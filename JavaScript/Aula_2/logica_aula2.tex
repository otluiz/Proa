%++++++% Preâmbulo %+++++++++++++++++++++++++++++++++++++++++++++++++++++++++
\documentclass[13pt, xcolor={dvipsnames,svgnames}, portuguese]{beamer}
%\documentclass[[11pt, xcolor={dvipsnames,svgnames,table},portuguese]{beamer} 

\usetheme{CambridgeUS}

\setbeamercolor*{structure}{bg=PineGreen!20,fg=PineGreen} %fg=PineGreen
\definecolor{beamer@pinegreen}{rgb}{0.137,0.666,0.741}


\setbeamercolor*{palette primary}{use=structure,fg=white,bg=structure.fg}
\setbeamercolor*{palette secondary}{use=structure,fg=white,bg=structure.fg!75}
\setbeamercolor*{palette tertiary}{use=structure,fg=white,bg=structure.fg!50!black}
\setbeamercolor*{palette quaternary}{fg=white,bg=black}

\setbeamercolor{section in toc}{fg=black,bg=white}
\setbeamercolor{alerted text}{use=structure,fg=structure.fg!50!black!80!black}

\setbeamercolor{titlelike}{parent=palette primary,fg=structure.fg!50!black}
\setbeamercolor{frametitle}{bg=gray!10!white,fg=PineGreen}

\setbeamercolor*{titlelike}{parent=palette primary}

\usepackage[utf8]{inputenc}
\usepackage[brazil]{babel}  % idioma
\usepackage{amsmath,amsfonts,amssymb,textcomp}
\usepackage{graphicx}
\usepackage{subfigure}
\usepackage[utf8]{inputenc}
\usepackage{ifpdf}
\usepackage{listings}

% Configurações para o ambiente lstlisting
\lstset{
    language=C,
    basicstyle=\ttfamily\footnotesize,
    numbers=left,
    numberstyle=\tiny,
    numbersep=5pt
}

% here you should include other packages with \usepackage

    \ifpdf

      % hyperref should be the last package loaded:
	  %\usepackage[pdftex]{hyperref}
      \usepackage{pst-pdf}
    \else

      % make the command \href from hyperref available as a 'print only'
      \newcommand{\href}[2]{#2}

    \fi

%Global Background must be put in preamble
\usebackgroundtemplate%
{%
    \includegraphics[width=\paperwidth,height=\paperheight]{Figuras/fundo.png}%
}
\setbeamertemplate{frametitle}[default][center]
 
\author{Othon Oliveira}
\title{Lógica de Programação com Java Script}
\institute{SENAC - PROA} 
\date{} 
%\subject{} 

\begin{document}

\begin{frame}
\titlepage
%\date{}
\end{frame}

% Capa - requer o TikZ
\newcommand{\capa}{
    \begin{tikzpicture}[remember picture,overlay]
        \node at (current page.south west)
            {\begin{tikzpicture}[remember picture, overlay]
                \fill[shading=radial,top color=orange,bottom color=orange,middle color=yellow] (0,0) rectangle (\paperwidth,\paperheight);
            \end{tikzpicture}
          };
    \end{tikzpicture}
}

\begin{frame}\frametitle{Sumário}
\tableofcontents
\end{frame}


%+++++++++++++++++++++++++++++++++++++++++++++++
\section{Entrada de dados}
%+++++++++++++++++++++++++++++++++++++++++++++++
\begin{frame}{Codando}
\framesubtitle{Introdução a Lógica de Programação com Java Script}
	\begin{block}{Leitura de dados através de prompt}
		\begin{itemize}
		  \item[a.] O Java Script trata a entrada de dados como tipo texto (string)
		  \pause
		  \item[b.] Uma maneira de converter os dados a entrada em número parseInt()
		   \pause		  
		  \item[c.] Essa função também trata se os dados não forem numéricos (NaN)
		  \pause
		  \item[d.] Dessa forma a entrada de dados já tem um tratamento 
		\end{itemize}
	\end{block} 
\end{frame}


\begin{frame}
\frametitle{JavaScript}
\framesubtitle{Introdução}

As funções parseInt() e parseFloat() são usadas para converter strings em valores numéricos, mas com algumas diferenças importantes.\\
 Além dessas, existem outras funções que também podem ser usadas para converter valores ou realizar outras operações.\\ 
 Vamos entender quando usar parseInt, parseFloat e outras funções relevantes:

%+++++++++++++++++++++++++++++++++++++++++++++++
\section{Funções Especiais}
%+++++++++++++++++++++++++++++++++++++++++++++++

\end{frame}

\begin{frame}
\frametitle{parseInt() - Conversão para Inteiro:}
\begin{itemize}
  \item Use parseInt() quando você quiser converter uma string em um valor inteiro
  \item Qualquer parte da string que não seja um número será ignorada
  \item Exemplo: parseInt("123") retorna 123
\end{itemize}
\end{frame}

\begin{frame}
\frametitle{parseFloat() - Conversão para Ponto Flutuante:}
\begin{itemize}
  \item Use parseFloat() quando você quiser converter uma string em um valor de ponto flutuante.
  \item Essa função lida com números decimais.
  \item Exemplo: parseFloat("3.14") retorna 3.14
\end{itemize}
\end{frame}

\begin{frame}
\frametitle{Number() - Conversão Genérica:}

\begin{itemize}
  \item A função Number() pode ser usada para converter tanto para inteiro quanto para ponto flutuante.
  \item Ela lida bem com números decimais e números em notação científica.
  \item Exemplo: Number("42") retorna 42, Number("3.14") retorna 3.14.
\end{itemize}
\end{frame}


\begin{frame}
\frametitle{isNaN() - Verificação de NaN:}

\begin{itemize}
  \item `isNaN()` verifica se um valor é NaN (Not-a-Number).
  \item Pode ser usado para validar se uma conversão de string para número foi bem-sucedida.
  \item Exemplo: isNaN(parseInt("abc")) retorna true.
\end{itemize}
\end{frame}


\begin{frame}
\frametitle{toFixed() - Arredondamento Decimal:}
\begin{itemize}
  \item A função toFixed() é usada para arredondar um número para um número específico de casas decimais.
  \item Exemplo: const numero = 3.14159; numero.toFixed(2) retorna "3.14".
\end{itemize}
\end{frame}

\begin{frame}
\frametitle{Math.floor() e Math.ceil() - Arredondamento para Baixo/Cima:}

\begin{itemize}
  \item Math.floor() arredonda um número para baixo para o inteiro mais próximo.
  \item Math.ceil() arredonda um número para cima para o inteiro mais próximo.
  \item Exemplo: Math.floor(4.8) retorna 4, Math.ceil(4.2) retorna 5.
\end{itemize}

\end{frame}


\begin{frame}
\frametitle{Math.random() - Geração de Números Aleatórios:}
\begin{itemize}
  \item Math.random() gera um número decimal aleatório entre 0 (inclusive) e 1 (exclusivo).
  \item Pode ser usado com outros métodos matemáticos para gerar números aleatórios em faixas específicas.
  \item Exemplo: Math.random() retorna um número aleatório entre 0 e 1.
\end{itemize}
\end{frame}


\begin{frame}[fragile]
\frametitle{Cálculos em JavaScript}
\framesubtitle{Usando Funções de Entrada}

\begin{verbatim}
// Exemplo 1: Soma de Dois Números
const num1 = parseFloat(prompt("Digite o primeiro número:"));
const num2 = parseFloat(prompt("Digite o segundo número:"));
const soma = num1 + num2;
console.log("A soma é:", soma);
\end{verbatim}
\end{frame}


\begin{frame}[fragile]

\begin{verbatim}
// Exemplo 3: Multiplicação de Números
const num1 = parseFloat(prompt("Digite o primeiro número:"));
const num2 = parseFloat(prompt("Digite o segundo número:"));
const multiplicacao = num1 * num2;
console.log("A multiplicação é:", multiplicacao);
\end{verbatim}

\end{frame}

%+++++++++++++++++++++++++++++++++++++++++++++++
\section{Calculos com um pouco mais desafio}
%+++++++++++++++++++++++++++++++++++++++++++++++

\begin{frame}[fragile]
\frametitle{Cálculos Desafiantes em JavaScript}
\framesubtitle{Desafio 1: Média Ponderada}

\begin{verbatim}
// Desafio: Calcular média ponderada de notas
const nota1 = parseFloat(prompt("Digite a nota 1:"));
const peso1 = parseFloat(prompt("Digite o peso da nota 1:"));
const nota2 = parseFloat(prompt("Digite a nota 2:"));
const peso2 = parseFloat(prompt("Digite o peso da nota 2:"));

const mediaPonderada = (nota1 * peso1 + nota2 * peso2) / (peso1 + peso2);
console.log("Média ponderada:", mediaPonderada);
\end{verbatim}

\end{frame}

%+++++++++++++++++++++++++++++++++++++++++++++++
\begin{frame}[fragile]
\frametitle{Cálculos Desafiantes em JavaScript}
\framesubtitle{Desafio 2: Juros Compostos}

\begin{verbatim}
// Desafio: Calcular montante com juros compostos
const principal=parseFloat(prompt("Digite o valor principal:"));
const taxaDeJuros=parseFloat(prompt("Digite a taxa de juros (%):"));
const anos = parseInt(prompt("Digite o número de anos:"));

const montante = principal * Math.pow(1 + taxaDeJuros / 100, anos);
console.log("Montante com juros compostos:", montante.toFixed(2));
\end{verbatim}

\end{frame}


\begin{frame}[fragile]
\frametitle{Cálculos Desafiantes em JavaScript}
\framesubtitle{Desafio 3: Fatorial}

\begin{verbatim}
// Desafio: Calcular o fatorial de um número
const numero = parseInt(prompt("Digite um número inteiro:"));

let fatorial = 1;
for (let i = 1; i <= numero; i++) {
  fatorial *= i;
}

console.log(`Fatorial de ${numero} é ${fatorial}`);
\end{verbatim}

\end{frame}


%+++++++++++++++++++++++++++++++++++++++++++++++


%+++++++++++++++++++++++++++++++++++++++++++++++

%+++++++++++++++++++++++++++++++++++++++++++++++


%+++++++++++++++++++++++++++++++++++++++++++++++


%+++++++++++++++++++++++++++++++++++++++++++++++

%%+++++++++++++++++++++++++++++++++++++++++++++++


%+++++++++++++++++++++++++++++++++++++++++++++++




%-----------------------------------------------

\end{document}