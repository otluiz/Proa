%++++++% Preâmbulo %+++++++++++++++++++++++++++++++++++++++++++++++++++++++++
\documentclass[13pt, xcolor={dvipsnames,svgnames}, portuguese]{beamer}
%\documentclass[[11pt, xcolor={dvipsnames,svgnames,table},portuguese]{beamer} 

\usetheme{CambridgeUS}

\setbeamercolor*{structure}{bg=PineGreen!20,fg=PineGreen} %fg=PineGreen
\definecolor{beamer@pinegreen}{rgb}{0.137,0.666,0.741}


\setbeamercolor*{palette primary}{use=structure,fg=white,bg=structure.fg}
\setbeamercolor*{palette secondary}{use=structure,fg=white,bg=structure.fg!75}
\setbeamercolor*{palette tertiary}{use=structure,fg=white,bg=structure.fg!50!black}
\setbeamercolor*{palette quaternary}{fg=white,bg=black}

\setbeamercolor{section in toc}{fg=black,bg=white}
\setbeamercolor{alerted text}{use=structure,fg=structure.fg!50!black!80!black}

\setbeamercolor{titlelike}{parent=palette primary,fg=structure.fg!50!black}
\setbeamercolor{frametitle}{bg=gray!10!white,fg=PineGreen}

\setbeamercolor*{titlelike}{parent=palette primary}

\usepackage[utf8]{inputenc}
\usepackage[brazil]{babel}  % idioma
\usepackage{amsmath,amsfonts,amssymb,textcomp}
\usepackage{graphicx}
\usepackage{subfigure}
\usepackage[utf8]{inputenc}
\usepackage{ifpdf}
\usepackage{listings}

% Configurações para o ambiente lstlisting
\lstset{
    language=C,
    basicstyle=\ttfamily\footnotesize,
    numbers=left,
    numberstyle=\tiny,
    numbersep=5pt
}

% here you should include other packages with \usepackage

    \ifpdf

      % hyperref should be the last package loaded:
	  %\usepackage[pdftex]{hyperref}
      \usepackage{pst-pdf}
    \else

      % make the command \href from hyperref available as a 'print only'
      \newcommand{\href}[2]{#2}

    \fi

%Global Background must be put in preamble
\usebackgroundtemplate%
{%
    \includegraphics[width=\paperwidth,height=\paperheight]{Figuras/fundo.png}%
}
\setbeamertemplate{frametitle}[default][center]
 
\author{Othon Oliveira}
\title{Lógica de Programação com Java Script}
\institute{SENAC - PROA} 
\date{} 
%\subject{} 

\begin{document}

\begin{frame}
\titlepage
%\date{}
\end{frame}

% Capa - requer o TikZ
\newcommand{\capa}{
    \begin{tikzpicture}[remember picture,overlay]
        \node at (current page.south west)
            {\begin{tikzpicture}[remember picture, overlay]
                \fill[shading=radial,top color=orange,bottom color=orange,middle color=yellow] (0,0) rectangle (\paperwidth,\paperheight);
            \end{tikzpicture}
          };
    \end{tikzpicture}
}

\begin{frame}\frametitle{Sumário}
\tableofcontents
\end{frame}


%+++++++++++++++++++++++++++++++++++++++++++++++
\section{Entrada de dados}
%+++++++++++++++++++++++++++++++++++++++++++++++
\begin{frame}{Introdução a Lógica de Programação com Java Script}
\framesubtitle{ Entrada de dados em Java Script}
	\begin{block}{Leitura de dados através de prompt}
		\begin{itemize}
		  \item[a.] O Java Script trata a entrada de dados como tipo texto (string)
		  \pause
		  \item[b.] Uma maneira de converter os dados a entrada em número: parseInt()
		   \pause		  
		  \item[c.] A função também trata se os dados não forem numéricos: (NaN)
		  \pause
		  \item[d.] Dessa forma, mostrada acima, a entrada de dados já tem um tratamento especial
		\end{itemize}
	\end{block} 
\end{frame}


%+++++++++++++++++++++++++++++++++++++++++++++++
\section{Funções Especiais}
%+++++++++++++++++++++++++++++++++++++++++++++++

\begin{frame}[fragile]
\frametitle{Cálculos em JavaScript}
\framesubtitle{Usando Funções de Entrada}

\begin{verbatim}
// Exemplo 1: Soma de Dois Números
const num1 = parseFloat(prompt("Digite o primeiro número:"));
const num2 = parseFloat(prompt("Digite o segundo número:"));
const soma = num1 + num2;
console.log("A soma é:", soma);
\end{verbatim}
Como se executa (rodar) esse código?
\end{frame}
%-------------------------------------------------------------
\begin{frame}[fragile]

\begin{verbatim}
// Exemplo 3: Multiplicação de Números
const num1 = parseFloat(prompt("Digite o primeiro número:"));
const num2 = parseFloat(prompt("Digite o segundo número:"));
const multiplicacao = num1 * num2;
console.log("A multiplicação é:", multiplicacao);
\end{verbatim}
Como se executa (rodar) esse código?
\end{frame}
%+++++++++++++++++++++++++++++++++++++++++++++++
\section{Calculos com um pouco de desafios}
%+++++++++++++++++++++++++++++++++++++++++++++++
\begin{frame}
\frametitle{Apresentando a Média Aritmética}

A média aritmética é uma medida de centralidade que representa o valor médio de um conjunto de números.

\vspace{0.5cm}

Para calcular a média aritmética:

\begin{enumerate}
\item Some todos os valores do conjunto.
\item Divida a soma pelo número total de valores no conjunto.
\end{enumerate}

\vspace{0.5cm}
Isso é expresso pela fórmula:

\[
\text{Média Aritmética} = \frac{v_1 + v_2 + v_3 + \ldots + v_N}{N}
\]

onde $v_i$ é cada valor individual do conjunto e $N$ é o número total de valores.
\end{frame}
%-------------------------------------------------------------
\begin{frame}[fragile]
\frametitle{Exemplo de Uso da Média Aritmética}
Vamos calcular a média aritmética de um conjunto de valores.
\begin{verbatim}
Vamos calcular a média aritmética dos valores 10, 15, 20, 25 e 30.

\begin{verbatim}
const valor1 = 10, valor2 = 15, valor3 = 20;
const valor4 = 25, valor5 = 30;

const soma = valor1 + valor2 + valor3 + valor4 + valor5;
const quantidade = 5;
const media = soma / quantidade;

console.log("Soma:", soma);
console.log("Média Aritmética:", media);

\end{verbatim}
Pergunta: Onde deverei "colar" esse código, para que funcione
do jeito que está?
\end{frame}

%+++++++++++++++++++++++++++++++++++++++++++++++
\section{Calculos com mais um pouco de desafios}
%+++++++++++++++++++++++++++++++++++++++++++++++
\begin{frame}
\frametitle{Apresentando a Média Ponderada}
A média ponderada é uma medida de centralidade que leva em consideração diferentes pesos atribuídos a cada valor no cálculo da média.
\vspace{0.5cm}
Para calcular a média ponderada:

\begin{enumerate}
\item Multiplique cada valor pelo seu respectivo peso.
\item Some os produtos obtidos.
\item Divida a soma pelo total dos pesos.
\end{enumerate}

\vspace{0.5cm}

Isso é expresso pela fórmula:
\[
\text{Média Ponderada} = \frac{v_1 \cdot p_1 + v_2 \cdot p_2 + v_3 \cdot p_3 + \ldots}{p_1 + p_2 + p_3 + \ldots}
\]
onde $v_i$ é cada valor individual do conjunto e $p_i$ é o peso correspondente a esse valor.

\end{frame}
%-------------------------------------------------------------
\begin{frame}[fragile]
\frametitle{Exemplo de Cálculo de Juros Simples}

Aqui está um exemplo de cálculo de juros simples em HTML:

\begin{verbatim}
<!DOCTYPE html>
<html>
<head> <title>Cálculo de Juros Simples</title> </head>
<body>
  <h1>Calculadora de Juros Simples</h1>
  <script>
    const principal = 5000;
    const taxaDeJuros = 10; // 10%
    const periodo = 2;
    const montante = principal + (principal * taxaDeJuros / 100 * periodo);
    const resultado = `Montante com Juros Simples: $${montante.toFixed(2)}`;
    document.write(resultado);
  </script>
</body>
</html>
\end{verbatim}

\end{frame}
%-------------------------------------------------------------
\begin{frame}[fragile]
\frametitle{Exemplo de Cálculo de Juros Simples}
Aqui está um exemplo de cálculo de juros simples em HTML com prompts de comando:

\begin{verbatim}
<!DOCTYPE html>
<html><head>
  <title>Cálculo de Juros Simples</title></head>
<body>
  <h1>Calculadora de Juros Simples</h1>
  <script>
    const principal = parseFloat(prompt("Digite o valor principal:"));
    const taxaDeJuros = parseFloat(prompt("Digite a taxa de juros (%):"));
    const periodo = parseFloat(prompt("Digite o período (anos):"));
    const montante = principal + (principal * taxaDeJuros / 100 * periodo);
    const resultado = `Montante com Juros Simples: $${montante.toFixed(2)}`;
    document.write(resultado);
  </script>
</body>
</html>
\end{verbatim}

\end{frame}

%%+++++++++++++++++++++++++++++++++++++++++++++++


%+++++++++++++++++++++++++++++++++++++++++++++++




%-----------------------------------------------

\end{document}