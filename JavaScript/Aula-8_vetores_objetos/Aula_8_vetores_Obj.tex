%++++++% Preâmbulo %+++++++++++++++++++++++++++++++++++++++++++++++++++++++++
\documentclass[13pt, xcolor={dvipsnames,svgnames}, portuguese]{beamer}
%\documentclass[[11pt, xcolor={dvipsnames,svgnames,table},portuguese]{beamer} 

\usetheme{CambridgeUS}

\setbeamercolor*{structure}{bg=PineGreen!20,fg=PineGreen} %fg=PineGreen
\definecolor{beamer@pinegreen}{rgb}{0.137,0.666,0.741}


\setbeamercolor*{palette primary}{use=structure,fg=white,bg=structure.fg}
\setbeamercolor*{palette secondary}{use=structure,fg=white,bg=structure.fg!75}
\setbeamercolor*{palette tertiary}{use=structure,fg=white,bg=structure.fg!50!black}
\setbeamercolor*{palette quaternary}{fg=white,bg=black}

\setbeamercolor{section in toc}{fg=black,bg=white}
\setbeamercolor{alerted text}{use=structure,fg=structure.fg!50!black!80!black}

\setbeamercolor{titlelike}{parent=palette primary,fg=structure.fg!50!black}
\setbeamercolor{frametitle}{bg=gray!10!white,fg=PineGreen}

\setbeamercolor*{titlelike}{parent=palette primary}

\usepackage[utf8]{inputenc}
\usepackage[brazil]{babel}  % idioma
\usepackage{amsmath,amsfonts,amssymb,textcomp}
\usepackage{graphicx}
\usepackage{subfigure}
\usepackage[utf8]{inputenc}
\usepackage{ifpdf}
\usepackage{listings}

% Configurações para o ambiente lstlisting
\lstset{
    language=C,
    basicstyle=\ttfamily\footnotesize,
    numbers=left,
    numberstyle=\tiny,
    numbersep=5pt
}

% here you should include other packages with \usepackage

    \ifpdf

      % hyperref should be the last package loaded:
	  %\usepackage[pdftex]{hyperref}
      \usepackage{pst-pdf}
    \else

      % make the command \href from hyperref available as a 'print only'
      \newcommand{\href}[2]{#2}

    \fi

%Global Background must be put in preamble
\usebackgroundtemplate%
{%
    \includegraphics[width=\paperwidth,height=\paperheight]{Figuras/fundo.png}%
}
\setbeamertemplate{frametitle}[default][center]
 
\author{Othon Oliveira}
\title{Lógica de Programação com Java Script}
\institute{SENAC - PROA} 
\date{} 
%\subject{} 

\begin{document}

\begin{frame}
\titlepage
%\date{}
\end{frame}

% Capa - requer o TikZ
\newcommand{\capa}{
    \begin{tikzpicture}[remember picture,overlay]
        \node at (current page.south west)
            {\begin{tikzpicture}[remember picture, overlay]
                \fill[shading=radial,top color=orange,bottom color=orange,middle color=yellow] (0,0) rectangle (\paperwidth,\paperheight);
            \end{tikzpicture}
          };
    \end{tikzpicture}
}

\begin{frame}\frametitle{Sumário}
\tableofcontents
\end{frame}


%+++++++++++++++++++++++++++++++++++++++++++++++
\section{Vetores}
%+++++++++++++++++++++++++++++++++++++++++++++++
\begin{frame}
\frametitle{Arrays e Vetores em JavaScript}
\begin{itemize}
  \item<1-> \textbf{Arrays e Vetores} são estruturas de dados usadas para armazenar coleções de elementos em JavaScript.
  \item<2-> Um \textbf{array} é uma estrutura de dados que pode conter elementos de qualquer tipo, acessados por índices numéricos.
  \item<3-> Um \textbf{vetor} é um array que possui elementos do mesmo tipo, otimizado para operações matemáticas.
\end{itemize}
\end{frame}

\begin{frame}
\frametitle{Criando Arrays e Vetores}
\begin{itemize}
  \item<1-> Em JavaScript, você pode criar um array usando colchetes: \texttt{let myArray = []}.
  \item<2-> Para criar um vetor, você pode usar bibliotecas como \texttt{NumPy} em Python ou \texttt{TypedArray} em JavaScript.
\end{itemize}
\end{frame}

\begin{frame}
\frametitle{Acessando Elementos}
\begin{itemize}
  \item<1-> Para acessar um elemento em um array, use a notação de colchetes: \texttt{myArray[0]}.
  \item<2-> Vetores geralmente possuem operações otimizadas para acessar elementos.
\end{itemize}
\end{frame}

\begin{frame}
\frametitle{Propriedades dos Arrays}
\begin{itemize}
  \item<1-> \texttt{length}: Retorna o número de elementos no array.
  \item<2-> \texttt{push()}: Adiciona um elemento ao final do array.
  \item<3-> \texttt{pop()}: Remove e retorna o último elemento do array.
  \item<4-> \texttt{splice()}: Adiciona ou remove elementos em posições específicas.
\end{itemize}
\end{frame}

\begin{frame}
\frametitle{Outras Propriedades dos Arrays}
\begin{itemize}
  \item<1-> \texttt{join()}: Converte os elementos em uma string, separados por um delimitador.
  \item<2-> \texttt{concat()}: Combina dois ou mais arrays e retorna um novo.
  \item<3-> \texttt{slice()}: Retorna uma cópia de parte do array, sem modificar o original.
  \item<4-> \texttt{indexOf()}: Retorna o índice do primeiro elemento correspondente.
\end{itemize}
\end{frame}
%--------------------------------------------------

\begin{frame}[fragile]
\frametitle{Uma função que mostra os Elementos}
\begin{verbatim}
function imprimirArray(arr) {
  // O código da função vai aqui
}
\end{verbatim}
Chame a função
\begin{verbatim}
//cria um vetor
let meuArray = [1, 2, 3, 4, 5];

// chama a função, passando o array
imprimirArray(meuArray);

\end{verbatim}

\end{frame}

%--------------------------------------------------
\begin{frame}[fragile]
\frametitle{enfim a função}
\begin{verbatim}
function imprimirArray(arr) {
  // O código da função vai aqui
}
\end{verbatim}
Chame a função
\begin{verbatim}
function imprimirArray(arr) {
  for (let i = 0; i < arr.length; i++) {
    console.log(arr[i]);
  }
}

\end{verbatim}

\end{frame}
%----------------------------------------------------
%+++++++++++++++++++++++++++++++++++++++++++++++
\section{Exemplos de propriedades}
%+++++++++++++++++++++++++++++++++++++++++++++++
\begin{frame}[fragile]
\frametitle{Propriedade \texttt{length}}
Length: Retorna o tamanho (número de elementos) do array.
\begin{verbatim}
let frutas = ["maçã", "banana", "laranja"];
let tamanho = frutas.length;
console.log(`O tamanho do array é ${tamanho}`);
\end{verbatim}
\end{frame}


\begin{frame}[fragile]
\frametitle{Método \texttt{push()}}
Push(): Adiciona um ou mais elementos ao final do array
\begin{verbatim}
let frutas = ["maçã", "banana"];
frutas.push("laranja");
console.log(frutas); // ["maçã", "banana", "laranja"]
\end{verbatim}
\end{frame}



\begin{frame}[fragile]
\frametitle{Método \texttt{pop()}}
Pop(): Remove e retorna o último elemento do array.
\begin{verbatim}
let frutas = ["maçã", "banana", "laranja"];
let ultimaFruta = frutas.pop();
console.log(ultimaFruta); // "laranja"
console.log(frutas); // ["maçã", "banana"]
\end{verbatim}
\end{frame}


\begin{frame}[fragile]
\frametitle{Método \texttt{splice()}}
splice(): Este método é usado para alterar o conteúdo de um array, adicionando ou removendo elementos. Ele recebe três argumentos: o índice inicial (onde a operação deve começar), o número de elementos a serem removidos e, opcionalmente, os elementos que devem ser adicionados ao array
\begin{verbatim}
let frutas = ["maçã", "banana", "laranja"];
frutas.splice(1, 1, "uva", "pêra");
console.log(frutas); // ["maçã", "uva", "pêra", "laranja"]
\end{verbatim}
\end{frame}





\begin{frame}[fragile]
\frametitle{Método \texttt{join()}}
Join(): Converte todos os elementos do array em uma única string, separados por um delimitador especificado
\begin{verbatim}
let frutas = ["maçã", "banana", "laranja"];
let texto = frutas.join(", ");
console.log(texto); // "maçã, banana, laranja"
\end{verbatim}
\end{frame}


\begin{frame}[fragile]
\frametitle{Método \texttt{concat()}}
Concat(): Combina dois ou mais arrays criando um novo array resultante
\begin{verbatim}
let frutas1 = ["maçã", "banana"];
let frutas2 = ["laranja", "uva"];
let frutasCombinadas = frutas1.concat(frutas2);
console.log(frutasCombinadas); // ["maçã", "banana", "laranja", "uva"]
\end{verbatim}
\end{frame}


\begin{frame}[fragile]
\frametitle{Método \texttt{slice()}}
slice(): Este método retorna uma parte de um array, definida por um índice inicial e um índice final. Ele não modifica o array original, mas cria um novo array com os elementos selecionados.
\begin{verbatim}
let frutas = ["maçã", "banana", "laranja", "uva"];
let fatia = frutas.slice(1, 3);
console.log(fatia); // ["banana", "laranja"]
\end{verbatim}
\end{frame}


\begin{frame}[fragile]
\frametitle{Método \texttt{indexOf()}}
IndexOf(): Retorna o primeiro índice em que um elemento especificado pode ser encontrado no array, ou -1 se o elemento não estiver presente
\begin{verbatim}
let frutas = ["maçã", "banana", "laranja"];
let indice = frutas.indexOf("banana");
console.log(`O índice de "banana" é ${indice}`);
\end{verbatim}
\end{frame}




% Slides dos exercícios

\begin{frame}[fragile]
\frametitle{Exercício 1: Soma dos Elementos}
\begin{verbatim}
// Dado um array de números, calcule a soma deles.
function somaArray(arr) {
  let soma = 0;
  for (let i = 0; i < arr.length; i++) {
    soma += arr[i];
  }
  return soma;
}
\end{verbatim}
\end{frame}
%---------------------------------------------
\begin{frame}[fragile]
\frametitle{Exercício 2: Encontrar o Maior Elemento}
\begin{verbatim}
// Encontre o maior elemento em um array.
function maiorElemento(arr) {
  let maior = arr[0];
  for (let i = 1; i < arr.length; i++) {
    if (arr[i] > maior) {
      maior = arr[i];
    }
  }
  return maior;
}
\end{verbatim}
\end{frame}

\begin{frame}[fragile]
\frametitle{Exercício 3: Filtrar Números Pares}
\begin{verbatim}
// Dado um array de números, filtre apenas os números pares.
function numerosPares(arr) {
  return arr.filter(numero => numero % 2 === 0);
}
\end{verbatim}
\end{frame}

% Adicione mais slides para os exercícios restantes...


%+++++++++++++++++++++++++++++++++++++++++++++++
\section{Objetos}
%+++++++++++++++++++++++++++++++++++++++++++++++
\begin{frame}[fragile]
\frametitle{Objetos}
Em JavaScript, um objeto é uma coleção de pares chave-valor, onde cada chave (também chamada de propriedade) está associada a um valor. Os objetos são usados para representar estruturas de dados complexas e são uma das principais características da linguagem.
\end{frame}



\begin{frame}[fragile]
\frametitle{Objetos}
Sintaxe: Os objetos são criados usando chaves \{\}. As propriedades e seus valores são definidos dentro das chaves, separados por dois-pontos :. As propriedades são separadas por vírgulas
\end{frame}

%+++++++++++++++++++++++++++++++++++++++++++++++
\section{Exemplos de objetos}
%+++++++++++++++++++++++++++++++++++++++++++++++

\begin{frame}[fragile]
\frametitle{Exemplo}
\begin{verbatim}
let pessoa = {
  nome: "João",
  idade: 30,
  cidade: "São Paulo"
};

\end{verbatim}
\end{frame}

%-------------------------------------------------------------

\begin{frame}[fragile]
\frametitle{Exemplo}
Acesso a Propriedades: Você pode acessar as propriedades de um objeto usando a notação de ponto (objeto.propriedade) ou a notação de colchetes (objeto['propriedade']).
\begin{verbatim}
console.log(pessoa.nome); // "João"
console.log(pessoa['idade']); // 30
\end{verbatim}
\end{frame}
%-------------------------------------------------------------


%%+++++++++++++++++++++++++++++++++++++++++++++++


%+++++++++++++++++++++++++++++++++++++++++++++++




%-----------------------------------------------

\end{document}
