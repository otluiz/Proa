%++++++% Preâmbulo %+++++++++++++++++++++++++++++++++++++++++++++++++++++++++
\documentclass[13pt, xcolor={dvipsnames,svgnames}, portuguese]{beamer}
%\documentclass[[11pt, xcolor={dvipsnames,svgnames,table},portuguese]{beamer} 

\usetheme{CambridgeUS}

\setbeamercolor*{structure}{bg=PineGreen!20,fg=PineGreen} %fg=PineGreen
\definecolor{beamer@pinegreen}{rgb}{0.137,0.666,0.741}


\setbeamercolor*{palette primary}{use=structure,fg=white,bg=structure.fg}
\setbeamercolor*{palette secondary}{use=structure,fg=white,bg=structure.fg!75}
\setbeamercolor*{palette tertiary}{use=structure,fg=white,bg=structure.fg!50!black}
\setbeamercolor*{palette quaternary}{fg=white,bg=black}

\setbeamercolor{section in toc}{fg=black,bg=white}
\setbeamercolor{alerted text}{use=structure,fg=structure.fg!50!black!80!black}

\setbeamercolor{titlelike}{parent=palette primary,fg=structure.fg!50!black}
\setbeamercolor{frametitle}{bg=gray!10!white,fg=PineGreen}

\setbeamercolor*{titlelike}{parent=palette primary}

\usepackage[utf8]{inputenc}
\usepackage[brazil]{babel}  % idioma
\usepackage{amsmath,amsfonts,amssymb,textcomp}
\usepackage{graphicx}
\usepackage{subfigure}
\usepackage[utf8]{inputenc}
\usepackage{ifpdf}
\usepackage{listings}

% Configurações para o ambiente lstlisting
\lstset{
    language=C,
    basicstyle=\ttfamily\footnotesize,
    numbers=left,
    numberstyle=\tiny,
    numbersep=5pt
}

% here you should include other packages with \usepackage

    \ifpdf

      % hyperref should be the last package loaded:
	  %\usepackage[pdftex]{hyperref}
      \usepackage{pst-pdf}
    \else

      % make the command \href from hyperref available as a 'print only'
      \newcommand{\href}[2]{#2}

    \fi

%Global Background must be put in preamble
\usebackgroundtemplate%
{%
    \includegraphics[width=\paperwidth,height=\paperheight]{Figuras/fundo.png}%
}
\setbeamertemplate{frametitle}[default][center]
 
\author{Othon Oliveira}
\title{Lógica de Programação com Java Script}
\institute{SENAC - PROA} 
\date{} 
%\subject{} 

\begin{document}

\begin{frame}
\titlepage
%\date{}
\end{frame}

% Capa - requer o TikZ
\newcommand{\capa}{
    \begin{tikzpicture}[remember picture,overlay]
        \node at (current page.south west)
            {\begin{tikzpicture}[remember picture, overlay]
                \fill[shading=radial,top color=orange,bottom color=orange,middle color=yellow] (0,0) rectangle (\paperwidth,\paperheight);
            \end{tikzpicture}
          };
    \end{tikzpicture}
}

\begin{frame}\frametitle{Sumário}
\tableofcontents
\end{frame}

%+++++++++++++++++++++++++++++++++++++++++++++++
\section{Apresentação}

%+++++++++++++++++++++++++++++++++++++++++++++++
\begin{frame}{Apresentação}
\framesubtitle{Quem sou  e objetivo da aula}
Othon L. T de Oliveira
	\begin{block}{Professor:}
		\begin{itemize}
		 \pause
		 \item Ensina: Introdução a Programação, Lógica de Programação, e outras disciplinas
		\end{itemize}
	\end{block}	
	\pause
	\begin{block}{Trabalho: }
		 Analista de Infraestrutura: Reitoria da UPE
	\end{block}
	\pause
	\begin{block}{Formação acadêmica}
		Engenharia Informática de Gestão (Politécnico de Portugal)
		\pause
		Mestrado em Engenharia de Sistemas (Escola Politécnica de Pernambuco - Poli)
		\pause
		Doutorando em Engenharia de Produção (Universidade Tecnilógica Federal do Paraná - UTFPR)
	\end{block}	

\end{frame}

%+++++++++++++++++++++++++++++++++++++++++++++++
\section{Introdução}
%+++++++++++++++++++++++++++++++++++++++++++++++
\begin{frame}{Conceitos}
\framesubtitle{Introdução a Lógica de Programação com Java Script}
	\begin{block}{Breve introdução}
		\begin{itemize}
		  \item[a.] Breve explicação sobre a importância da lógica de programação 
		  \pause
		  \item[b.] Como ela é usada em todos os tipos de desenvolvimento de software
		   \pause		  
		  \item[c.] Java Script é basicamente uma linguagem client-side (navegadores)
		  \pause
		  \item[d.] Contudo Java Script hoje em dia é também server-side (servidores)
		\end{itemize}
	\end{block} 
\end{frame}


\begin{frame}
\frametitle{JavaScript}
\framesubtitle{Introdução}

JavaScript é uma linguagem de programação amplamente utilizada para desenvolvimento web.

\begin{itemize}
  \item Interpretada, executada no navegador
  \item Adiciona interatividade e dinamismo às páginas web
  \item Compatível com HTML e CSS
\end{itemize}

\end{frame}

\begin{frame}
\frametitle{JavaScript}
\framesubtitle{Recursos}

Recursos Principais:

\begin{itemize}
  \item Manipulação do DOM
  \item Eventos e interatividade
  \item Requisições assíncronas (AJAX)
  \item Frameworks populares: React, Angular, Vue
\end{itemize}

Vantagens:

\begin{itemize}
  \item Amplamente suportado nos navegadores
  \item Versatilidade e flexibilidade
  \item Comunidade ativa e vasta documentação
\end{itemize}
\end{frame}

\begin{frame}
\frametitle{DOM - Modelo de Objeto de Documento}

O DOM é uma representação em árvore da estrutura do documento HTML.

\begin{itemize}
  \item Permite manipular elementos HTML com JavaScript
  \item Facilita adicionar, remover ou alterar conteúdo
  \item Atualizações dinâmicas em páginas web
\end{itemize}

\end{frame}

\begin{frame}
\frametitle{Ajax - Asynchronous JavaScript and XML}

Ajax é uma técnica que permite atualizações assíncronas em páginas web.

\begin{itemize}
  \item Atualiza partes da página sem recarregá-la completamente
  \item Melhora a experiência do usuário e economiza largura de banda
  \item Usa requisições HTTP assíncronas para buscar ou enviar dados
\end{itemize}

\end{frame}

\begin{frame}
\frametitle{React}

React é uma biblioteca JavaScript para construir interfaces de usuário.

\begin{itemize}
  \item Desenvolvido pelo Facebook
  \item Componentização: construção de UIs em pedaços reutilizáveis
  \item Atualizações eficientes com a virtual DOM
  \item Uso em aplicações web e mobile (React Native)
\end{itemize}

\end{frame}


\begin{frame}
\frametitle{Vue.js}

Vue.js é um framework progressivo para construir interfaces de usuário.

\begin{itemize}
  \item Fácil integração com projetos existentes
  \item Abordagem gradual: adicione Vue a partes específicas do projeto
  \item Reativo e eficiente, com renderização de componentes
  \item Ecossistema sólido e documentação abrangente
\end{itemize}

\end{frame}


\begin{frame}
\frametitle{JavaScript}
\framesubtitle{Lado do Servidor}

Hoje em dia, JavaScript também é utilizado no lado do servidor, com o Node.js.

\begin{itemize}
  \item Node.js permite executar JavaScript no servidor
  \item Ideal para desenvolvimento de aplicações escaláveis e em tempo real
  \item Facilita a criação de servidores web e APIs
\end{itemize}

\end{frame}
%+++++++++++++++++++++++++++++++++++++++++++++++
\section{Operadores lógicos}
%+++++++++++++++++++++++++++++++++++++++++++++++

\begin{frame}{Operadores Lógicos}
\framesubtitle{Operador Lógico AND (\&\&)}
	\begin{block}{Atuação}
		\begin{itemize}
		  \item[a. ] O operador lógico AND (\&\&) combina duas condições 
		  \pause
		  \item[b. ] Retorna verdadeiro (true) apenas se ambas as condições forem verdadeiras
		\end{itemize}
	\end{block}	
\end{frame}


%+++++++++++++++++++++++++++++++++++++++++++++++
\begin{frame}[fragile]
\framesubtitle{Operador Lógico AND (\&\&)}
 Exemplo de pseudo-código usando o operador lógico AND

	\begin{verbatim}
		SE (condição1 for verdadeira) E (condição2 for verdadeira)
		ENTAO
		  // Executar código se ambas as condições forem verdadeiras
		FIM SE
	\end{verbatim}
	---------------------------------\\
	Exemplo em Java Script
	\begin{verbatim}
	if (condicao1 && condicao2) {
  		// Executar código se ambas as condições forem verdadeiras
	}

	\end{verbatim}
\end{frame}


%+++++++++++++++++++++++++++++++++++++++++++++++
\begin{frame}{Operadores Lógicos}
\framesubtitle{Operador Lógico OR ($||$)}
	\begin{block}{Atuação}
		\begin{itemize}
		  \item[a. ] O operador lógico OR ($||$) combina duas condições 
		  \pause
		  \item[b. ] Retorna verdadeiro (true) se apenas uma das condições for verdadeira
		\end{itemize}
	\end{block}	
\end{frame}

%+++++++++++++++++++++++++++++++++++++++++++++++
\begin{frame}[fragile]
\framesubtitle{Operador Lógico OR ($||$)}
 Exemplo de pseudo-código usando o operador lógico OR
	\begin{verbatim}
	SE (condição1 for verdadeira) OU (condição2 for verdadeira)
	ENTAO
  // Executar código se pelo menos uma das condições for verdadeira
	FIM SE
	\end{verbatim}
	---------------------------------\\
	Exemplo em Java Script
	\begin{verbatim}
	if (condicao1 || condicao2) {
  	// Executar código se pelo menos uma das condições for verdadeira
	}
	\end{verbatim}
\end{frame}

%+++++++++++++++++++++++++++++++++++++++++++++++
\begin{frame}{Operadores Lógicos}
\framesubtitle{Operador Lógico NOT ($!$)}
	\begin{block}{Atuação}
		\begin{itemize}
		  \item[a. ] O operador lógico NOT ($!$) inverte o valor de uma condição, transformando verdadeiro em falso e falso em verdadeiro. 
		\end{itemize}
	\end{block}	
\end{frame}

%+++++++++++++++++++++++++++++++++++++++++++++++
\begin{frame}[fragile]
\framesubtitle{Operador Lógico NOT ($!$)}
 Exemplo de pseudo-código usando o operador lógico NOT
	\begin{verbatim}
	SE NAO (condição for verdadeira)
	ENTAO
  	// Executar código se a condição for falsa
	FIM SE
	\end{verbatim}
	---------------------------------\\
	Exemplo em Java Script
	\begin{verbatim}
	if (!condicao) {
  	// Executar código se a condição for falsa
	}
	\end{verbatim}
\end{frame}


%+++++++++++++++++++++++++++++++++++++++++++++++
\begin{frame}[fragile]
\framesubtitle{Exemplo prático de um programa Java Script}
 Demonstração de um exemplo prático de uso de operadores lógicos em JavaScript.\\
 Código Java Script:
	\begin{verbatim}
	const idade = 15;
    const possuiPermissao = true;

	if (idade >= 18 || possuiPermissao) {
  		console.log("Acesso permitido!");
	} else {
  		console.log("Acesso negado!");
	}
	\end{verbatim}	
	Qual o resultado, após executar o código?
\end{frame}

%%+++++++++++++++++++++++++++++++++++++++++++++++
\section{Estruturas de decisão}
\subsection*{IF - ELSE}
%%+++++++++++++++++++++++++++++++++++++++++++++++
\begin{frame}
  \frametitle{Introdução às Estruturas de Decisão}
  Breve explicação sobre a importância das estruturas de decisão em programação e como elas permitem que o programa tome decisões com base em condições.

\end{frame}

%+++++++++++++++++++++++++++++++++++++++++++++++
\begin{frame}[fragile]
  \frametitle{Estrutura Básica de um Programa}
  \begin{itemize}
    \item Exemplo de Pseudo-Linguagem:
    \begin{verbatim}
  SE (condição for verdadeira)
  ENTAO
    // Executar código se a condição for verdadeira
  FIM SE
    \end{verbatim}
  \end{itemize}
  
   \begin{itemize}
    \item Exemplo Java Script:
    \begin{verbatim}
  if (condicao) {
    // Executar código se a condição for verdadeira
  }
    \end{verbatim}
  \end{itemize}
\end{frame}


\begin{frame}[fragile]
  \frametitle{Estrutura IF-ELSE IF-ELSE}
  A estrutura \texttt{if-else if-else} permite avaliar várias condições em sequência e executar o bloco de código correspondente à primeira condição verdadeira.
  \newline
  Exemplo em Pseudo-Linguagem:
  \begin{verbatim}
  SE (condição1 for verdadeira)
  ENTAO
    // Executar código se a condição1 for verdadeira
  SENAO SE (condição2 for verdadeira)
    // Executar código se a condição2 for verdadeira
  SENAO
    // Executar código se nenhuma das condições for verdadeira
  FIM SE
  \end{verbatim}
\end{frame}

\begin{frame}[fragile]
  \frametitle{Estrutura IF-ELSE IF-ELSE}
  A estrutura \texttt{if-else if-else} permite avaliar várias condições em sequência e executar o bloco de código correspondente à primeira condição verdadeira.
  \newline
  Exemplo em JavaScript:
  \begin{verbatim}
  if (condicao1) {
    // Executar código se a condição1 for verdadeira
  } else if (condicao2) {
    // Executar código se a condição2 for verdadeira
  } else {
    // Executar código se nenhuma das condições for verdadeira
  }
  \end{verbatim}
\end{frame}

\begin{frame}[fragile]
  \frametitle{Estrutura SWITCH-CASE}
  A estrutura \texttt{switch-case} é usada para avaliar um valor e executar um bloco de código correspondente ao valor.
  \newline
  Exemplo em Pseudo-Linguagem:
  \begin{verbatim}
  ESCOLHA (valor)
  CASO opcao1:
    // Executar código se o valor for opcao1
    INTERROMPER
  CASO opcao2:
    // Executar código se o valor for opcao2
    INTERROMPER
  PADRAO:
    // Executar código se o valor não for correspondente a nenhum caso
  FIM ESCOLHA
  \end{verbatim}
\end{frame}

\begin{frame}[fragile]
  \frametitle{Estrutura SWITCH-CASE}
  A estrutura \texttt{switch-case} é usada para avaliar um valor e executar um bloco de código correspondente ao valor.
  \newline
  Exemplo em JavaScript:
  \begin{verbatim}
  switch (valor) {
    case opcao1:
      // Executar código se o valor for opcao1
      break;
    case opcao2:
      // Executar código se o valor for opcao2
      break;
    default:
      // Executar código se o valor não for correspondente a nenhum caso
  }
  \end{verbatim}
\end{frame}


\begin{frame}[fragile]
  \frametitle{Exemplo Prático em JavaScript (IF-ELSE)}
  %\newline
  Exemplo em JavaScript:
  \begin{verbatim}
  const idade = 15;

  if (idade >= 18) {
    console.log("Você é maior de idade.");
  } else {
    console.log("Você é menor de idade.");
  }
  \end{verbatim}
\end{frame}


\begin{frame}[fragile]
  \frametitle{Exemplo Prático em JavaScript (IF-ELSE IF-ELSE)}
  Código JavaScript:
  \begin{verbatim}
  const nota = 85;

  if (nota >= 90) {
    console.log("Nota A");
  } else if (nota >= 80) {
    console.log("Nota B");
  } else if (nota >= 70) {
    console.log("Nota C");
  } else {
    console.log("Nota abaixo de C");
  }
  \end{verbatim}
  Com base na nota, é exibida a mensagem correspondente à faixa de notas.
\end{frame}


\begin{frame}[fragile]
  \frametitle{Exemplo Prático em JavaScript (SWITCH-CASE)}
  Código JavaScript:
  \begin{verbatim}
  const diaDaSemana = "quarta";
  switch (diaDaSemana) {
    case "segunda":
      console.log("Dia de trabalho"); break;
    case "terca":
      console.log("Dia de trabalho"); break;
    case "quarta":
      console.log("Meio da semana");  break;
    case "quinta":
      console.log("Dia de trabalho"); break;
    case "sexta":
      console.log("Dia de trabalho"); break;
    default:
      console.log("Fim de semana");
  }
  \end{verbatim}
  Com base no dia da semana, é exibida a mensagem correspondente à descrição do dia.
\end{frame}


\begin{frame}[fragile]
  \frametitle{Conversão de Temperatura}
  Vamos pratica?\\ Faça um programinha que solicita ao usuário uma temperatura em graus Celsius e realiza a conversão para graus Fahrenheit e Kelvin.\\
  Há duas soluções apresentadas. Uma roda diretamente no console do navegador.\\
  A outra você deve fazer criar um arquivo HTML (com o nome que você escolher).\\
  O código HTML está em dois slides, contudo você deve colocar tudo em um só arquivo.\\
  Lembrando novamente, essa é uma solução possível. Você não deve se limitar ao código do professor.\\
  É muito mais importante você modificar o cógido, fazendo do seu jeito, ou, fazer um totalmente diferente.
\end{frame}

\begin{frame}[fragile]
  \frametitle{Conversão de Temperatura (Pseudo-Linguagem)}
  Pseudo-Linguagem: (ou pseudo-código)
  \begin{verbatim}
  LER temperaturaCelcius
  temperaturaFahrenheit <- (temperaturaCelcius * 9/5) + 32
  temperaturaKelvin <- temperaturaCelcius + 273.15
  ESCREVER "Temperatura em Fahrenheit:", temperaturaFahrenheit
  ESCREVER "Temperatura em Kelvin:", temperaturaKelvin
  \end{verbatim}
\end{frame}


\begin{frame}[fragile]
  \frametitle{Conversão de Temperatura (JavaScript)}
  Código JavaScript:
  \begin{verbatim}
  const temperaturaCelcius = parseFloat(prompt(
        "Digite a temperatura em Celsius:"));
  const temperaturaFahrenheit = (temperaturaCelcius * 9/5) + 32;
  const temperaturaKelvin = temperaturaCelcius + 273.15;

  console.log("Temperatura em Fahrenheit:", temperaturaFahrenheit);
  console.log("Temperatura em Kelvin:", temperaturaKelvin);
  \end{verbatim}
  Esse código pode rodar diretamente na console do navegador (F12), por exemplo; no navegador Firefox ou no navegador Chrome.
\end{frame}



\begin{frame}[fragile]
  \frametitle{Conversão de Temperatura (HTML) - primeira parte}
  Código HTML:
  \begin{verbatim}
  <!DOCTYPE html>
<html lang="en">
<head>
  <meta charset="UTF-8">
  <meta name="viewport" content="width=device-width, 
     initial-scale=1.0">
  <title>Conversão de Temperatura</title>
</head>
<body>
  <h1>Conversão de Temperatura</h1>
  <label>Digite a temperatura em Celsius: <input type="number" 
    id="temperaturaInput"></label>
  <button onclick="converterTemperatura()">Converter</button>
  <p id="resultado"></p>
  \end{verbatim}
\end{frame}


\begin{frame}[fragile]
  \frametitle{Conversão de Temperatura (HTML) - segunda parte}
  Código HTML:
  \begin{verbatim}
  <script>
  function conversaoTemperatura() {
   const temperaturaCelsius = parseFloat(
   document.getElementById("temperaturaInput").value);
      
   const temperaturaFahrenheit=(temperaturaCelsius * 9/5) + 32;
   const temperaturaKelvin = temperaturaCelsius + 273.15;
   const resultado = `
     Temperatura em Celsius: ${temperaturaCelsius}°C<br>
     Temperatura em Fahrenheit: ${temperaturaFahrenheit.toFixed(2)}°F<br>
     Temperatura em Kelvin: ${temperaturaKelvin.toFixed(2)}K
   `;
  \end{verbatim}
\end{frame}

\begin{frame}[fragile]
  \frametitle{Conversão de Temperatura (HTML) - terceira parte}
  Código JavaScript:
  \begin{verbatim}

   document.getElementById("resultado").innerHTML = resultado;
  }
  </script>
</body>
</html>
  \end{verbatim}
 Terceira e última parte do código html. Juntar as 3 partes num só arquivo
\end{frame}




%-----------------------------------------------

\end{document}