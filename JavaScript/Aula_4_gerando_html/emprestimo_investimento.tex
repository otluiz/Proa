%++++++% Preâmbulo %+++++++++++++++++++++++++++++++++++++++++++++++++++++++++
\documentclass[13pt, xcolor={dvipsnames,svgnames}, portuguese]{beamer}
%\documentclass[[11pt, xcolor={dvipsnames,svgnames,table},portuguese]{beamer} 

\usetheme{CambridgeUS}

\setbeamercolor*{structure}{bg=PineGreen!20,fg=PineGreen} %fg=PineGreen
\definecolor{beamer@pinegreen}{rgb}{0.137,0.666,0.741}


\setbeamercolor*{palette primary}{use=structure,fg=white,bg=structure.fg}
\setbeamercolor*{palette secondary}{use=structure,fg=white,bg=structure.fg!75}
\setbeamercolor*{palette tertiary}{use=structure,fg=white,bg=structure.fg!50!black}
\setbeamercolor*{palette quaternary}{fg=white,bg=black}

\setbeamercolor{section in toc}{fg=black,bg=white}
\setbeamercolor{alerted text}{use=structure,fg=structure.fg!50!black!80!black}

\setbeamercolor{titlelike}{parent=palette primary,fg=structure.fg!50!black}
\setbeamercolor{frametitle}{bg=gray!10!white,fg=PineGreen}

\setbeamercolor*{titlelike}{parent=palette primary}

\usepackage[utf8]{inputenc}
\usepackage[brazil]{babel}  % idioma
\usepackage{amsmath,amsfonts,amssymb,textcomp}
\usepackage{graphicx}
\usepackage{subfigure}
\usepackage[utf8]{inputenc}
\usepackage{ifpdf}
\usepackage{listings}

% Configurações para o ambiente lstlisting
\lstset{
    language=C,
    basicstyle=\ttfamily\footnotesize,
    numbers=left,
    numberstyle=\tiny,
    numbersep=5pt
}

% here you should include other packages with \usepackage

    \ifpdf

      % hyperref should be the last package loaded:
	  %\usepackage[pdftex]{hyperref}
      \usepackage{pst-pdf}
    \else

      % make the command \href from hyperref available as a 'print only'
      \newcommand{\href}[2]{#2}

    \fi

%Global Background must be put in preamble
\usebackgroundtemplate%
{%
    \includegraphics[width=\paperwidth,height=\paperheight]{Figuras/fundo.png}%
}
\setbeamertemplate{frametitle}[default][center]
 
\author{Othon Oliveira}
\title{Lógica de Programação com Java Script}
\institute{SENAC - PROA} 
\date{} 
%\subject{} 

\begin{document}

\begin{frame}
\titlepage
%\date{}
\end{frame}

% Capa - requer o TikZ
\newcommand{\capa}{
    \begin{tikzpicture}[remember picture,overlay]
        \node at (current page.south west)
            {\begin{tikzpicture}[remember picture, overlay]
                \fill[shading=radial,top color=orange,bottom color=orange,middle color=yellow] (0,0) rectangle (\paperwidth,\paperheight);
            \end{tikzpicture}
          };
    \end{tikzpicture}
}

\begin{frame}\frametitle{Sumário}
\tableofcontents
\end{frame}


%+++++++++++++++++++++++++++++++++++++++++++++++
\section{Entrada de dados}
%+++++++++++++++++++++++++++++++++++++++++++++++
\begin{frame}{Introdução a Lógica de Programação com Java Script}
\framesubtitle{ Entrada de dados em Java Script}
	\begin{block}{Leitura de dados através de prompt}
		\begin{itemize}
		  \item[a.] O Java Script trata a entrada de dados como tipo texto (string)
		  \pause
		  \item[b.] Contudo, após receber os dados pode ser que precisemos mudar o tipo.
		   \pause		  
		  \item[c.] Para cada caso (do tipo de dado) presisamos tratar dentro do algoritmo. 
		  \pause
		  \item[d.] Então, como e quando devemos mudar os tipos de dados que trabalhamos no algoritmo?
		\end{itemize}
	\end{block} 
\end{frame}


%+++++++++++++++++++++++++++++++++++++++++++++++
\section{Variáveis e Constantes em JavaScript}
%+++++++++++++++++++++++++++++++++++++++++++++++
\begin{frame}[fragile]
\frametitle{Variáveis \texttt{let} e \texttt{const}}

Na linguagem JavaScript, usamos \texttt{let} e \texttt{const} para declarar variáveis.

\begin{itemize}
  \item \texttt{let}: Permite criar variáveis mutáveis.
  \item \texttt{const}: Cria variáveis imutáveis (constantes).
\end{itemize}

Exemplo:
\begin{verbatim}
const valorEmprestado = 2000;
let resultado = "";
\end{verbatim}

\end{frame}

%+++++++++++++++++++++++++++++++++++++++++++++++
\section{Calculos com um pouco de desafios}
%+++++++++++++++++++++++++++++++++++++++++++++++
\begin{frame}
João é uma pessoa que gosta de emprestar dinheiro aos amigos, contudo,
\pause 
chegando próximo ao final do ano João descobriu que seu dinheiro acabou.
\pause
João então teve que pedir dinheiro emprestado para fechar as contas no final do ano.
\pause
Será que João teve prezuízo?
\end{frame}
%-------------------------------------------------------------
\begin{frame}
O valor total que foi emprestado a outrem foi de 2.000 (dois mil reais)
\pause 
a uma taxa de 10\% ao mês,
\pause
o valor que João foi pedido a um agiota foi de 1900 (um mil e novecentos reais)
\pause
a uma taxa de 11\% ao mês.
\pause
 No final de um ano João ganhou ou perdeu dinheiro?
\end{frame}
%-------------------------------------------------------------
\begin{frame}
\frametitle{Algoritmo: Empréstimo X Investimento}

Vamos fazer um algoritmo calcula se uma pessoa ganhou ou perdeu dinheiro após um ano.
\vspace{0.5cm}
\begin{enumerate}
  \item Defina as constantes para os valores de empréstimo, taxas e juros.
  \item Calcule os juros de empréstimo e valor pedido emprestado a alguém.
  \item Calcule o saldo final.
  \item Verifique se houve lucro, prejuízo ou equilíbrio.
  \item Exiba o resultado no console.
\end{enumerate}

\end{frame}
%-------------------------------------------------------------
\begin{frame}[fragile]
\frametitle{Código do Algoritmo}

\begin{verbatim}
// Dados dos empréstimos e investimentos
const valorEmprestado = 2000;
const taxaEmprestimo = 0.10;
const valorPedidoEmprestado = 1900;
const taxaPedidoEmprestado = 0.11;

// Cálculos de juros...

// Cálculo do saldo final...

// Verificação do resultado...

// Exibir o resultado...
\end{verbatim}

\end{frame}

%-------------------------------------------------------------
\begin{frame}[fragile]
\frametitle{Exemplo de Cálculo de Juros Simples}

Aqui está um exemplo de cálculo de juros simples em HTML:

\begin{verbatim}
<!DOCTYPE html>
<html>
<head> <title>Cálculo de Juros Simples</title> </head>
<body>
  <h1>Calculadora de Juros Simples</h1>
  <script>
    const principal = 5000;
    const taxaDeJuros = 10; // 10%
    const periodo = 2;
    const montante = principal + (principal * taxaDeJuros / 100 * periodo);
    const resultado = `Montante com Juros Simples: $${montante.toFixed(2)}`;
    document.write(resultado);
  </script>
</body>
</html>
\end{verbatim}

\end{frame}
%-------------------------------------------------------------
\begin{frame}[fragile]
\frametitle{Exemplo de Cálculo de Juros Simples}
Aqui está um exemplo de cálculo de juros simples em HTML com prompts de comando:

\begin{verbatim}
<!DOCTYPE html>
<html><head>
  <title>Cálculo de Juros Simples</title></head>
<body>
  <h1>Calculadora de Juros Simples</h1>
  <script>
    const principal = parseFloat(prompt("Digite o valor principal:"));
    const taxaDeJuros = parseFloat(prompt("Digite a taxa de juros (%):"));
    const periodo = parseFloat(prompt("Digite o período (anos):"));
    const montante = principal + (principal * taxaDeJuros / 100 * periodo);
    const resultado = `Montante com Juros Simples: $${montante.toFixed(2)}`;
    document.write(resultado);
  </script>
</body>
</html>
\end{verbatim}

\end{frame}
%-------------------------------------------------------------
\begin{frame}[fragile]
\frametitle{Continuação ..}

\begin{verbatim}
//continuação
// Dados iniciais
// Dados dos empréstimos e investimentos
const valorEmprestado = 2000; // Valor total emprestado (R$)
const taxaEmprestimo = 0.10; // (10% ao mês)
const valorPedidoEmprestado = 1900; // Valor pedido emprestado (R$)
const taxaPedidoEmprestado = 0.11; // (11% ao mês)
// Cálculo dos juros
// Juros de empréstimo em um ano (12 meses)
const jurosEmprestimo = valorEmprestado * taxaEmprestimo * 12; 
// Juros de empréstimo do valor pedido em um ano (12 meses)
const jurosPedidoEmprestado = valorPedidoEmprestado * taxaPedidoEmprestado * 12; 
// Cálculo do saldo final
const saldoFinal = jurosEmprestimo - jurosPedidoEmprestado;

\end{verbatim}

\end{frame}
%-------------------------------------------------------------
\begin{frame}[fragile]
\frametitle{Exemplo de Cálculo de Juros Simples - JavaScript}

\begin{verbatim}
// Dados finais
// Verificação do resultado
let resultado = "";
if (saldoFinal > 0) {
  resultado = "Lucro! ganhou dinheiro";
} else if (saldoFinal < 0) {
  resultado = "Prejuízo! perdeu dinheiro";
} else {
  resultado = "Equilíbrio! Zero de saldo final";
}

// Exibir o resultado
console.log(resultado);

\end{verbatim}

\end{frame}
%-------------------------------------------------------------

\end{document}