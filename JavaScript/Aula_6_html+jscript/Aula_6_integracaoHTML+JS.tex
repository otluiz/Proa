%++++++% Preâmbulo %+++++++++++++++++++++++++++++++++++++++++++++++++++++++++
\documentclass[13pt, xcolor={dvipsnames,svgnames}, portuguese]{beamer}
%\documentclass[[11pt, xcolor={dvipsnames,svgnames,table},portuguese]{beamer} 

\usetheme{CambridgeUS}

\setbeamercolor*{structure}{bg=PineGreen!20,fg=PineGreen} %fg=PineGreen
\definecolor{beamer@pinegreen}{rgb}{0.137,0.666,0.741}


\setbeamercolor*{palette primary}{use=structure,fg=white,bg=structure.fg}
\setbeamercolor*{palette secondary}{use=structure,fg=white,bg=structure.fg!75}
\setbeamercolor*{palette tertiary}{use=structure,fg=white,bg=structure.fg!50!black}
\setbeamercolor*{palette quaternary}{fg=white,bg=black}

\setbeamercolor{section in toc}{fg=black,bg=white}
\setbeamercolor{alerted text}{use=structure,fg=structure.fg!50!black!80!black}

\setbeamercolor{titlelike}{parent=palette primary,fg=structure.fg!50!black}
\setbeamercolor{frametitle}{bg=gray!10!white,fg=PineGreen}

\setbeamercolor*{titlelike}{parent=palette primary}

\usepackage[utf8]{inputenc}
\usepackage[brazil]{babel}  % idioma
\usepackage{amsmath,amsfonts,amssymb,textcomp}
\usepackage{graphicx}
\usepackage{subfigure}
\usepackage[utf8]{inputenc}
\usepackage{ifpdf}
\usepackage{listings}

% Configurações para o ambiente lstlisting
\lstset{
    language=C,
    basicstyle=\ttfamily\footnotesize,
    numbers=left,
    numberstyle=\tiny,
    numbersep=5pt
}

% here you should include other packages with \usepackage

    \ifpdf

      % hyperref should be the last package loaded:
	  %\usepackage[pdftex]{hyperref}
      \usepackage{pst-pdf}
    \else

      % make the command \href from hyperref available as a 'print only'
      \newcommand{\href}[2]{#2}

    \fi

%Global Background must be put in preamble
\usebackgroundtemplate%
{%
    \includegraphics[width=\paperwidth,height=\paperheight]{Figuras/fundo.png}%
}
\setbeamertemplate{frametitle}[default][center]
 
\author{Othon Oliveira}
\title{Lógica de Programação com Java Script}
\institute{SENAC - PROA} 
\date{} 
%\subject{} 

\begin{document}

\begin{frame}
\titlepage
%\date{}
\end{frame}

% Capa - requer o TikZ
\newcommand{\capa}{
    \begin{tikzpicture}[remember picture,overlay]
        \node at (current page.south west)
            {\begin{tikzpicture}[remember picture, overlay]
                \fill[shading=radial,top color=orange,bottom color=orange,middle color=yellow] (0,0) rectangle (\paperwidth,\paperheight);
            \end{tikzpicture}
          };
    \end{tikzpicture}
}

\begin{frame}\frametitle{Sumário}
\tableofcontents
\end{frame}


%+++++++++++++++++++++++++++++++++++++++++++++++
\section{Interação HTML X JAVA SCRIPT}
%+++++++++++++++++++++++++++++++++++++++++++++++
\begin{frame}
\frametitle{Integração entre HTML e JavaScript}

\begin{itemize}
  \item Botão Simples
  \item Funções Simples
  \item Interação Complexa
\end{itemize}

\end{frame}

\begin{frame}[fragile]
\frametitle{Botão Simples}

\begin{verbatim}
<!DOCTYPE html>
<html>
<head>
  <title>Integração HTML/JavaScript</title>
</head>
<body>
  <h1>Clique no botão</h1>
  <button onclick="mostrarMensagem()">Clique aqui</button>

  <script>
    function mostrarMensagem() {
      alert("Você clicou no botão!");
    }
  </script>
</body>
</html>
\end{verbatim}

\end{frame}


% Slides para as demais funções (3 a 11) continuam aqui...
%-----------------------------------------------
\begin{frame}[fragile]
\frametitle{Encapsular as Funções JavaScript no HTML}

\begin{itemize}
  \item Exercício 1: Somar Dois Números
  \item Exercício 2: Verificar Número Par
  \item Exercício 3: Calcular Média
  \item Exercício 4: Converter Polegadas para Centímetros
  \item Exercício 5: Verificar Triângulo
  \item Exercício 6: Calcular Área do Triângulo
  \item Exercício 7: Verificar Maior Número
  \item Exercício 8: Calcular Desconto
  \item Exercício 9: Verificar Número Primo
  \item Exercício 10: Calcular Fatorial
\end{itemize}

\end{frame}
%-----------------------------------------------
\begin{frame}[fragile]
\frametitle{Exercício 1: Função Somar Dois Números}

\begin{verbatim}
<!DOCTYPE html>
<html>
<head>
  <title>Integração HTML/JavaScript</title>
</head>
<body>
  <h1>Função Somar Dois Números</h1>
  <button onclick="somarNumeros()">Somar</button>

  <script>
    function somarNumeros() {
      const num1 = parseFloat(prompt("Digite o primeiro número:"));
      const num2 = parseFloat(prompt("Digite o segundo número:"));
      const resultado = num1 + num2;
      alert(`A soma é: ${resultado}`);
    }
  </script>
</body>
</html>
\end{verbatim}

\end{frame}

\begin{frame}[fragile]
\frametitle{Exercício 2: Função par ou impar}

\begin{verbatim}
<!DOCTYPE html>
<html>
<head>
  <title>Integração HTML/JavaScript</title>
</head>
<body>
  <h1>Função Par ou Ímpar</h1>
  <button onclick="parImpar()">Testar</button>

  <script>
    function parImpar() {
      const num1 = parseFloat(prompt("Digite o primeiro número:"));
	  if(num1 % 2 == 0){
      	alert(`É par o número: ${num1}`);
      else{
      	alert(`É ímpar o número: ${num1}`);};
    }
  </script>
</body>
</html>
\end{verbatim}

\end{frame}

%-----------------------------------------------
\begin{frame}[fragile]
\frametitle{Funções como Parte de Expressões - SUGESTÃO}
Função sem nome
\begin{verbatim}
const saudacao = function(nome) {
  return `Olá, ${nome}!`;
};

const mensagem = saudacao("Eva");
console.log(mensagem); // Saída: Olá, Eva!
\end{verbatim}

\end{frame}

%-----------------------------------------------
\begin{frame}[fragile]
\frametitle{Encapsulando a função anterior - SUGESTÃO}

\begin{verbatim}
<!DOCTYPE html>
<html>
<head>
  <title>Integração HTML/JavaScript</title>
</head>
<body>
  <h1>Mostrar mensagem de saudação</h1>
  <!-- Adicione um campo de entrada de texto -->
  <input type="text" id="nome">
  <button onclick="exibirSaudacao()">Saudação</button>

  <script>
    const saudacao = function(nome) {
      return `Olá, ${nome}!`;
    };
\end{verbatim}

\end{frame}
%-----------------------------------------------
\begin{frame}[fragile]
\frametitle{Encapsulando a função anterior - SUGESTÃO}

\begin{verbatim}

    // Função para exibir a saudação com base no valor do campo de entrada
    function exibirSaudacao() {
      const nome = document.getElementById("nome").value;
      const mensagem = saudacao(nome);
      alert(mensagem); // Exibir a mensagem em um alerta
    }
  </script>
</body>
</html>
\end{verbatim}

\end{frame}

%--------------------------------------------------------
\begin{frame}[fragile]
\frametitle{Mudando estilos com a tab style}

\begin{verbatim}
<!DOCTYPE html>
<html>
<head>
  <title>Exemplo de CSS com a tag style</title>
  <style>
    /* Definição de estilos */
    body {
      font-family: Arial, sans-serif;
      background-color: #f0f0f0;
    }
    h1 {
      color: blue;
    }
 
\end{verbatim}

\end{frame}
%-------------------------------------------------------
\begin{frame}[fragile]
\frametitle{Mudando estilos com a tab style}

\begin{verbatim}

    p {
      font-size: 16px;
      margin-bottom: 10px;
    }
  </style>
</head>
<body>
  <h1>Título da Página</h1>
  <p>Este é um parágrafo com estilo definido.</p>
</body>
</html>
\end{verbatim}

\end{frame}

%-----------------------------------------------

\begin{frame}[fragile]
\frametitle{Carregando o CSS em arquivo separado}

\begin{verbatim}
<!DOCTYPE html>
<html>
<head>
  <title>Exemplo de CSS com arquivo separado</title>
  <link rel="stylesheet" type="text/css" href="estilos.css">
</head>
<body>
  <h1>Título da Página</h1>
  <p>Este é um parágrafo com estilo definido.</p>
</body>
</html>
\end{verbatim}

\end{frame}
%-------------------------------------------------------
\begin{frame}[fragile]
\frametitle{Ajustanto para duas colunas - controle do foco}

\begin{verbatim}
<!DOCTYPE html>
<html>
<head>
  <title>Integração HTML/JavaScript</title>
  <style>
    .column {
      float: left;
      width: 50%;
    }
    .clear {
      clear: both;
    }
  </style>
</head>
<body>
  <h1>Controle de Foco por Ação</h1>
\end{verbatim}

\end{frame}

%-----------------------------------------------


\begin{frame}[fragile]
\frametitle{Ajustanto foco em duas colunas - continuação}

\begin{verbatim}
  <div class="column">
    <button onclick="definirFoco('input2')">
    			Preencher Campo 1</button>
    <input type="text" id="input1" placeholder="Campo 1">
  </div>
  <div class="column">
    <button onclick="definirFoco('input1')">
    			Preencher Campo 2</button>
    <input type="text" id="input2" placeholder="Campo 2">
  </div>
  <script>
    function definirFoco(elementId) {
      document.getElementById(elementId).focus();
    }
  </script></body></html>
\end{verbatim}

\end{frame}
%-----------------------------------------------
% Slides anteriores...

\begin{frame}[fragile]
\frametitle{Controle de Foco por Ação - tudo em uma coluna}

\begin{columns}
\column{0.5\textwidth}
\begin{verbatim}
<!DOCTYPE html>
<html>
<head>
  <title>Integração HTML/JavaScript</title>
  <style>
    .column {
      width: 100%;
    }
  </style>
</head>
<body>
  <h1>Controle de Foco por Ação</h1>
\end{verbatim}

\column{0.5\textwidth}
\end{columns}

\end{frame}
%--------------------------------------------------------

\begin{frame}[fragile]
\frametitle{Controle de Foco por Ação - tudo em uma coluna}

\begin{columns}
\column{0.5\textwidth}
\begin{verbatim}
  <div class="column">
    <button onclick="definirFoco('input1')">Preencher Campo 1</button>
    <input type="text" id="input1" placeholder="Campo 1"><br><br>
    <button onclick="definirFoco('input2')">Preencher Campo 2</button>
    <input type="text" id="input2" placeholder="Campo 2">
  </div>

  <script>
    function definirFoco(elementId) {
      document.getElementById(elementId).focus();
    }
  </script>
</body>
</html>
\end{verbatim}

\column{0.5\textwidth}
\end{columns}

\end{frame}



% Slides para os exemplos restantes (20 a 22) continuam aqui...

\end{document}



%-----------------------------------------------

%-----------------------------------------------


%-----------------------------------------------

%-----------------------------------------------

\end{document}