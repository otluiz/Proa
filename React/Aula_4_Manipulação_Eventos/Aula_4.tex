%++++++% Preâmbulo %+++++++++++++++++++++++++++++++++++++++++++++++++++++++++
\documentclass[13pt, xcolor={dvipsnames,svgnames}, portuguese]{beamer}

\usetheme{CambridgeUS}
\setbeamercolor*{structure}{bg=PineGreen!20,fg=PineGreen} %fg=PineGreen
\definecolor{beamer@pinegreen}{rgb}{0.137,0.666,0.741}


\setbeamercolor*{palette primary}{use=structure,fg=white,bg=structure.fg}
\setbeamercolor*{palette secondary}{use=structure,fg=white,bg=structure.fg!75}
\setbeamercolor*{palette tertiary}{use=structure,fg=white,bg=structure.fg!50!black}
\setbeamercolor*{palette quaternary}{fg=white,bg=black}

\setbeamercolor{section in toc}{fg=black,bg=white}
\setbeamercolor{alerted text}{use=structure,fg=structure.fg!50!black!80!black}

\setbeamercolor{titlelike}{parent=palette primary,fg=structure.fg!50!black}
\setbeamercolor{frametitle}{bg=gray!10!white,fg=PineGreen}

\setbeamercolor*{titlelike}{parent=palette primary}

\usepackage[utf8]{inputenc}
\usepackage[brazil]{babel}  % idioma
\usepackage{amsmath,amsfonts,amssymb,textcomp}
\usepackage{graphicx}
\usepackage{subfigure}
\usepackage[utf8]{inputenc}
\usepackage{ifpdf}
\usepackage{listings}

% Configurações para o ambiente lstlisting
\lstset{
    language=C,
    basicstyle=\ttfamily\footnotesize,
    numbers=left,
    numberstyle=\tiny,
    numbersep=5pt
}

\lstdefinelanguage{JavaScript}{
  keywords={typeof, new, true, false, catch, function, return, null, catch, switch, var, if, in, while, do, else, case, break},
  keywordstyle=\color{blue}\bfseries,
  ndkeywords={class, export, boolean, throw, implements, import, this},
  ndkeywordstyle=\color{darkgray}\bfseries,
  identifierstyle=\color{black},
  sensitive=false,
  comment=[l]{//},
  morecomment=[s]{/*}{*/},
  commentstyle=\color{purple}\ttfamily,
  stringstyle=\color{red}\ttfamily,
  morestring=[b]',
  morestring=[b]"
}

% here you should include other packages with \usepackage
    \ifpdf

      % hyperref should be the last package loaded:
	  %\usepackage[pdftex]{hyperref}
      \usepackage{pst-pdf}
    \else

      % make the command \href from hyperref available as a 'print only'
      \newcommand{\href}[2]{#2}

    \fi

%Global Background must be put in preamble
\usebackgroundtemplate%
{%
    \includegraphics[width=\paperwidth,height=\paperheight]{Figuras/fundo.png}%
}
\setbeamertemplate{frametitle}[default][center]
 
\author{Othon Oliveira}
\title{Introdução ao React.JS}
\institute{SENAC - PROA} 
\date{} 
%\subject{} 

\begin{document}

\begin{frame}
\titlepage
%\date{}
\end{frame}

% Capa - requer o TikZ
\newcommand{\capa}{
    \begin{tikzpicture}[remember picture,overlay]
        \node at (current page.south west)
            {\begin{tikzpicture}[remember picture, overlay]
                \fill[shading=radial,top color=orange,bottom color=orange,middle color=yellow] (0,0) rectangle (\paperwidth,\paperheight);
            \end{tikzpicture}
          };
    \end{tikzpicture}
}

\begin{frame}\frametitle{Sumário}
\tableofcontents
\end{frame}


\begin{frame}
\frametitle{Minha Imagem}
\begin{figure}
\centering
\includegraphics[width=0.5\textwidth]{Figuras/react1.png}
\caption{Componentes React}
\end{figure}
\end{frame}

%+++++++++++++++++++++++++++++++++++++++++++++++
\section{Respondendo a eventos no React}
%+++++++++++++++++++++++++++++++++++++++++++++++
\begin{frame}{O que são eventos?}
\framesubtitle{Um simples click no botão}
	\begin{block}{Handle event}
       Você pode responder a eventos declarando funções de manipulador de eventos (handle event) dentro de seus componentes:
	\end{block}	
	
\end{frame}


\begin{frame}[fragile]{Exemplo}
\frametitle{ClickComponente.js}
\begin{lstlisting}[language=JavaScript]
function MeuBotao() {
  function manipulaClick() {
    alert('Voce clicou em mim!');
  }

  return (
    <button onClick={manipulaClick}>
      Click em mim
    </button>
  );
}
\end{lstlisting}

\begin{alertblock}{Bora fazer?}
	Observe como onClick=\{manipulaClick\} não tem parênteses no final! Não chame a função do manipulador de eventos: você só precisa transmiti-la. O React chamará seu manipulador de eventos quando o usuário clicar no botão.
\end{alertblock}
\end{frame}

%++++++++++++++++++++++++++++++++++++++++++++++++++++++++++
\section{Ganchos ou Hooks?}
%++++++++++++++++++++++++++++++++++++++++++++++++++++++++++
\begin{frame}{O que é um Hook?}
\framesubtitle{Traduzindo ao pé da letra o termo}
	\begin{block}{Gancho}
       Hook: Um dispositivo curvado, metálico, na forma de um gancho;
       Hook também pode ser um termo usado nas lutas de boxe, gancho de esquerda (ou direita): a left/rigth hook
	\end{block}
\pause
	\begin{block}{Hook em React}
       Os hooks (ganchos) em React são funções que permitem "conectar-se" ao estado do React e aos recursos do ciclo de vida a partir de componentes.
	\end{block}
\pause
\begin{figure}
%\centering
\includegraphics[width=0.13\textwidth]{Figuras/hookCaptain.png}
\caption{A volta do Capitão Gancho}
\end{figure}
\end{frame}


\begin{frame}{Tipos de hooks}
\framesubtitle{Existem dois tipos principais}
	\begin{block}{useState }
       Este hook (gancho) permite adicionar estado aos seus componentes funcionais. Ele pega um valor de estado inicial e retorna um array com o valor do estado atual e uma função para atualizar o estado.
	\end{block}	
	\pause
	\begin{block}{useEffect }
       Este gancho permite que você execute efeitos ``colaterais'', como busca de dados, quando seu componente é montado ou atualizado.
	\end{block}		
	
\end{frame}
%---------------------------------------------------------
\begin{frame}[fragile]{Criando o componente Hook}
\framesubtitle{Mundaça de estado -\> Um simples click no botão}
	\begin{block}{Handle event}
       primeiro vamos importar ``useState'' do React
	\end{block}	
	
\begin{lstlisting}[language=JavaScript]
 import { useState } from 'react';
\end{lstlisting}
Agora você pode declarar uma variável de estado dentro do seu componente:

\begin{lstlisting}[language=JavaScript]
function MeuBotao() {
  const [contar, setContar] = useState(0);
  // ...
}
\end{lstlisting}
\end{frame}
%---------------------------------------------------------

\begin{frame}[fragile]
\frametitle{Quem são os manipuladores?}
\framesubtitle{[``algumaCoisa'', ``setAlgumaCoisa'']}

Na primeira vez que o botão for exibido, a contagem será 0 porque você passou 0 para useState(). Quando você quiser mudar de estado, chame setContar() e passe o novo valor para ele. Clicar neste botão aumentará o contador:


\begin{lstlisting}[language=JavaScript]
function MeuBotao() {
  const [contar, setContar] = useState(0);
  
  function manipulaClick() {
    // ...
  }
}
\end{lstlisting}

\end{frame}


%+++++++++++++++++++++++++++++++++++++++++++++++
\section{Adicionando ``estado''}
%+++++++++++++++++++++++++++++++++++++++++++++++

\begin{frame}[fragile]
\frametitle{Mundando o estado de um componente}
\framesubtitle{Bora fazer?}

Na primeira vez que o botão for exibido, a contagem será 0 porque você passou 0 para useState(). Quando você quiser mudar de estado, chame setContar() e passe o novo valor para ele. Clicar neste botão aumentará o contador:
\begin{lstlisting}[language=JavaScript]
function MeuBotao() {
  const [contar, setContar] = useState(0);

  function manipulaClick() {
    setContar(contar + 1);
  }

  return (
    <button onClick={manipulaClick}>
      Clicou {contar} vezes
    </button>
  );
}
\end{lstlisting}

\end{frame}
%+++++++++++++++++++++++++++++++++++++++++++++++
\section{Upgrade ao componente ContaBancaria}
%+++++++++++++++++++++++++++++++++++++++++++++++
\begin{frame}[fragile]
\frametitle{Upgrade}
\framesubtitle{Bora fazer?}

\begin{lstlisting}[language=JavaScript]
import React, { useState } from 'react';
import ContaBancaria from './ContaBancaria';

function App() {
  const [saldo, setSaldo] = useState(1000); // Valor inicial do saldo

  const atualizarSaldo = (novoSaldo) => {
    setSaldo(novoSaldo);
  };
  return (
    <div>
      <h1>Minha Aplicacao Bancria</h1>
      <ContaBancaria saldoInicial={saldo} atualizarSaldo={atualizarSaldo} />
    </div>
  );
}
export default App;
\end{lstlisting}

\end{frame}



%-------------------------------------------------------------


%+++++++++++++++++++++++++++++++++++++++++++++++%
%\section{Calculos com }
%+++++++++++++++++++++++++++++++++++++++++++++++

%-------------------------------------------------------------

%-------------------------------------------------------------


%%+++++++++++++++++++++++++++++++++++++++++++++++


%+++++++++++++++++++++++++++++++++++++++++++++++




%-----------------------------------------------

\end{document}
