%++++++% Preâmbulo %+++++++++++++++++++++++++++++++++++++++++++++++++++++++++
\documentclass[13pt, xcolor={dvipsnames,svgnames}, portuguese]{beamer}

\usetheme{CambridgeUS}

\setbeamercolor*{structure}{bg=PineGreen!20,fg=PineGreen} %fg=PineGreen
\definecolor{beamer@pinegreen}{rgb}{0.137,0.666,0.741}


\setbeamercolor*{palette primary}{use=structure,fg=white,bg=structure.fg}
\setbeamercolor*{palette secondary}{use=structure,fg=white,bg=structure.fg!75}
\setbeamercolor*{palette tertiary}{use=structure,fg=white,bg=structure.fg!50!black}
\setbeamercolor*{palette quaternary}{fg=white,bg=black}

\setbeamercolor{section in toc}{fg=black,bg=white}
\setbeamercolor{alerted text}{use=structure,fg=structure.fg!50!black!80!black}

\setbeamercolor{titlelike}{parent=palette primary,fg=structure.fg!50!black}
\setbeamercolor{frametitle}{bg=gray!10!white,fg=PineGreen}

\setbeamercolor*{titlelike}{parent=palette primary}

\usepackage[utf8]{inputenc}
\usepackage[brazil]{babel}  % idioma
\usepackage{amsmath,amsfonts,amssymb,textcomp}
\usepackage{graphicx}
\usepackage{subfigure}
\usepackage[utf8]{inputenc}
\usepackage{ifpdf}
\usepackage{listings}

% Configurações para o ambiente lstlisting
\lstset{
    language=C,
    basicstyle=\ttfamily\footnotesize,
    numbers=left,
    numberstyle=\tiny,
    numbersep=5pt
}

%\usepackage{xcolor}
%
%\lstdefinestyle{mystyle}{
%  language=JavaScript,
%  basicstyle=\ttfamily,
%  keywordstyle=\color{blue},
%  commentstyle=\color{green},
%  stringstyle=\color{red},
%  breakatwhitespace=false,
%  breaklines=true,
%  captionpos=b,
%  keepspaces=true,
%  numbers=left,
%  numbersep=5pt,
%  showspaces=false,
%  showstringspaces=false,
%  showtabs=false,
%  tabsize=2
%}
%\lstset{style=mystyle}



% here you should include other packages with \usepackage

    \ifpdf

      % hyperref should be the last package loaded:
	  %\usepackage[pdftex]{hyperref}
      \usepackage{pst-pdf}
    \else

      % make the command \href from hyperref available as a 'print only'
      \newcommand{\href}[2]{#2}

    \fi

%Global Background must be put in preamble
\usebackgroundtemplate%
{%
    \includegraphics[width=\paperwidth,height=\paperheight]{Figuras/fundo.png}%
}
\setbeamertemplate{frametitle}[default][center]
 
\author{Othon Oliveira}
\title{O ambiente React}
\institute{SENAC - PROA} 
\date{} 
%\subject{} 

\begin{document}

\begin{frame}
\titlepage
%\date{}
\end{frame}

% Capa - requer o TikZ
\newcommand{\capa}{
    \begin{tikzpicture}[remember picture,overlay]
        \node at (current page.south west)
            {\begin{tikzpicture}[remember picture, overlay]
                \fill[shading=radial,top color=orange,bottom color=orange,middle color=yellow] (0,0) rectangle (\paperwidth,\paperheight);
            \end{tikzpicture}
          };
    \end{tikzpicture}
}

\begin{frame}\frametitle{Sumário}
\tableofcontents
\end{frame}


%+++++++++++++++++++++++++++++++++++++++++++++++
\section{O ambiente React}
%+++++++++++++++++++++++++++++++++++++++++++++++

%------- slide 1 ----------------------
\begin{frame}{Introdução ao React.js}
  \title{Aula 1: Introdução ao React.js}
  \subtitle{Duração: 4 horas}
  \maketitle
\end{frame}

%------- slide 2 ----------------------
\begin{frame}{O que é React.js?}
  \begin{block}{Definição}
    React.js é uma biblioteca JavaScript de código aberto que permite a construção de interfaces de usuário interativas e reativas.
  \end{block}

  \begin{itemize}
    \item Desenvolvida e mantida pelo Facebook.
    \item Lançada em 2013.
    \item Ampla adoção na indústria de desenvolvimento web.
  \end{itemize}
\end{frame}

%------- slide 3 ----------------------
\begin{frame}[fragile]
 \framesubtitle{Por que usar o React?}
  \begin{block}{Vantagens do React}
    \begin{itemize}
      \item Reatividade: Atualizações automáticas quando os dados mudam.
      \item Componentização: Interface dividida em componentes reutilizáveis.
      \item Virtual DOM: Melhora o desempenho das atualizações de interface.
    \end{itemize}
  \end{block}
  
\begin{exampleblock}{Exemplo de Componente}
    \begin{verbatim}
	function Button(props) {
  		return <button>{props.label}<button>;
	}
    \end{verbatim}
\end{exampleblock}
\end{frame}


%------- slide 4 ----------------------
\begin{frame}[fragile]
\framesubtitle{Configurando o Ambiente}
  \begin{block}{Passos para Configurar o Ambiente}
    \begin{enumerate}
      \item Instale o Node.js e o npm (ou yarn) no seu sistema.
      \item Abra o terminal ou prompt de comando.
      \item Crie um novo diretório para o seu projeto:
      
      \begin{verbatim}
      mkdir meu-projeto-react
      cd meu-projeto-react
      \end{verbatim}
      
      \item Inicialize um novo projeto React usando o Create React App:
      
      \begin{verbatim}
      npx create-react-app .
      \end{verbatim}
      
      \item Aguarde a instalação das dependências.
      \item Inicie o servidor de desenvolvimento:
      
      \begin{verbatim}
      npm start
      \end{verbatim}
      
      \item Seu ambiente de desenvolvimento React está pronto!
    \end{enumerate}
  \end{block}
\end{frame}

\section{Componentes}
%------ slide 5 ------------------------------
\begin{frame}{Criando um Componente}
  \begin{itemize}
    \item Um componente é uma parte isolada da interface.
    \item Exemplo: um componente de "Botão".
  \end{itemize}
\end{frame}

%------ slide 6 ------------------------------
\begin{frame}{Estrutura de um Componente}
  \begin{block}{Estrutura Básica de um Componente}
    Um componente em React pode ser definido como uma função ou uma classe. Ambas as abordagens têm uma estrutura básica comum.
  \end{block}
\end{frame}

%------ slide 6.1 ----------------------------
\begin{frame}[fragile]{Estrutura de um Componente (Função)}
  \begin{exampleblock}{Exemplo de Componente como Função}
    \begin{verbatim}
function MeuComponente(props) {
  return (
    <div>
      <h1>{props.titulo}</h1>
      <p>{props.texto}</p>
    </div>
  );
}
    \end{verbatim}
  \end{exampleblock}
\end{frame}

%------ slide 6.2 ----------------------------
\begin{frame}[fragile]{Estrutura de um Componente (Classe)}
  \begin{exampleblock}{Exemplo de Componente como Classe}
    \begin{verbatim}
class MeuComponente extends React.Component {
  render() {
    return (
      <div>
        <h1>{this.props.titulo}</h1>
        <p>{this.props.texto}</p>
      </div>
    );
  }
}
    \end{verbatim}
  \end{exampleblock}
\end{frame}

%------ slide 6.3-1 ----------------------------
\begin{frame}{Diferenças entre Função e Classe (1/3)}
  \begin{block}{Componente como Função}
    - Declaração simples como uma função.\\
    - Usa `props` como argumento.\\
    - Não mantém estado interno (state).
  \end{block}
\end{frame}

%----- slide 6.3-2---------------------------
\begin{frame}{Diferenças entre Função e Classe (2/3)}
  \begin{block}{Componente como Classe}
    - Declaração como uma classe que estende `React.Component`.\\
    - Usa `this.props` para acessar as propriedades.\\
    - Pode manter estado interno (state) com `this.state`.
  \end{block}
\end{frame}

%------- slide 6.3-3 ----------------------------------

\begin{frame}{Diferenças entre Função e Classe (3/3)}
  \begin{block}{Escolhendo entre Função e Classe}
    - Use componentes como funções quando possível para simplicidade.\\
    - Use componentes como classes quando precisa de estado interno.\\
    - React 16.8+ oferece Hooks para adicionar estado a componentes de função.
  \end{block}
\end{frame}


\section{Estado Interno de um componente}
%------------ slide 7 ------------------------------------
\begin{frame}{Estado Interno (Parte 1)}
  \begin{block}{O que é Estado Interno?}
    O estado interno (state) em um componente React é uma forma de armazenar e gerenciar dados que podem mudar ao longo do tempo.\\ É usado para rastrear informações específicas ao componente que podem afetar a renderização.
  \end{block}
\end{frame}

%------------ slide 7.1 ------------------------------------
\begin{frame}[fragile]{Estado Interno (Função)}
  \begin{block}{Uso de Estado Interno em Componentes como Função}
    Em componentes como função, você pode usar o Hook `useState` para adicionar estado interno.
  \end{block}
  
  \begin{exampleblock}{Exemplo de Componente como Função com Estado Interno}
    \begin{verbatim}
import React, { useState } from 'react';
function Contador() {
  const [contador, setContador] = useState(0);
  return (
   <div>
    <p>Contagem: {contador}</p>
    <button onClick={() => setContador(contador + 1)}>
    Incrementar </button>
    </div>
  );
}
    \end{verbatim}
  \end{exampleblock}
\end{frame}

%------------ slide 7.2 ------------------------------------
\begin{frame}[fragile]{Estado Interno (Classe)}
  \begin{block}{Uso de Estado Interno em Componentes como Classe}
    Em componentes como classe, você pode definir o estado interno usando `this.state` e atualizá-lo com `this.setState`.
  \end{block}
  
  \begin{exampleblock}{Exemplo de Componente como Classe com Estado Interno}
    \begin{verbatim}
class Contador extends React.Component {
  constructor(props) {
    super(props);
    this.state = { contador: 0 };
  }
  render() {
 // próximo slide
    \end{verbatim}
  \end{exampleblock}
\end{frame}


%------------ slide 7.2 ------------------------------------
\begin{frame}[fragile]{Estado Interno (Classe)}
  \begin{block}{Uso de Estado Interno em Componentes como Classe}
    Em componentes como classe, você pode definir o estado interno usando `this.state` e atualizá-lo com `this.setState`.
  \end{block}
  
  \begin{exampleblock}{Exemplo de Componente como Classe com Estado Interno}
    \begin{verbatim}
// continua abaixo do render()
   return (
    <div>
     <p>Contagem: {this.state.contador}</p>
      <button onClick={() => this.setState({ contador: this.state.contador + 1 })}>Incrementar
      </button>
    </div>
    ); 
  }
}
    \end{verbatim}
  \end{exampleblock}
\end{frame}

%------------ slide 7.3 ------------------------------------
%------------ slide 7.4 ------------------------------------


%------ slide 7 ------------------------------
\begin{frame}{JSX (JavaScript XML)}
  \begin{itemize}
    \item Sintaxe semelhante a HTML no JavaScript.
    \pause
    \item Permite a criação de elementos de interface de forma declarativa.
  \end{itemize}
\end{frame}

%------ slide 8 ------------------------------
\begin{frame}[fragile]{Renderização de Componentes}
  \begin{itemize}
    \item Usando o ReactDOM.render() para renderizar um componente em um elemento HTML existente na página.
  \end{itemize}
  
  \begin{exampleblock}{Renderizando um Componente}
    \begin{verbatim}
    import React from 'react';
    import ReactDOM from 'react-dom';
    
    const element = <Button label="Clique-me" />;
    
    ReactDOM.render(element, document.getElementById('root'));
    \end{verbatim}
  \end{exampleblock}
\end{frame}

%------ slide 9 ------------------------------
\begin{frame}[fragile]{Atividade Prática}
  \begin{itemize}
    \item Exercício prático: Crie um componente React simples e renderize-o em uma página web.
  \end{itemize}
  
\begin{exampleblock}{Código de um Componente "HelloWorld" React}
    \begin{verbatim}
    import React from 'react';
    import ReactDOM from 'react-dom';
    
    function HelloWorld() {
      return <h1>Hello, React!</h1>;
    }
    
    ReactDOM.render(<HelloWorld />, document.getElementById('root'));
    \end{verbatim}
\end{exampleblock}
\end{frame}
%------ slide 10 -----------------------------


%%+++++++++++++++++++++++++++++++++++++++++++++++


%+++++++++++++++++++++++++++++++++++++++++++++++




%-----------------------------------------------

\end{document}