%++++++% Preâmbulo %+++++++++++++++++++++++++++++++++++++++++++++++
\documentclass[13pt, xcolor={dvipsnames,svgnames}, portuguese]{beamer}

\usetheme{CambridgeUS}

\setbeamercolor*{structure}{bg=PineGreen!20,fg=PineGreen} %fg=PineGreen
\definecolor{beamer@pinegreen}{rgb}{0.137,0.666,0.741}


\setbeamercolor*{palette primary}{use=structure,fg=white,bg=structure.fg}
\setbeamercolor*{palette secondary}{use=structure,fg=white,bg=structure.fg!75}
\setbeamercolor*{palette tertiary}{use=structure,fg=white,bg=structure.fg!50!black}
\setbeamercolor*{palette quaternary}{fg=white,bg=black}

\setbeamercolor{section in toc}{fg=black,bg=white}
\setbeamercolor{alerted text}{use=structure,fg=structure.fg!50!black!80!black}

\setbeamercolor{titlelike}{parent=palette primary,fg=structure.fg!50!black}
\setbeamercolor{frametitle}{bg=gray!10!white,fg=PineGreen}

\setbeamercolor*{titlelike}{parent=palette primary}

\usepackage[utf8]{inputenc}
\usepackage[brazil]{babel}  % idioma
\usepackage{amsmath,amsfonts,amssymb,textcomp}
\usepackage{graphicx}
\usepackage{subfigure}
\usepackage[utf8]{inputenc}
\usepackage{ifpdf}
\usepackage{listings}

% Configurações para o ambiente lstlisting
\lstset{
    language=C,
    basicstyle=\ttfamily\footnotesize,
    numbers=left,
    numberstyle=\tiny,
    numbersep=5pt
}

\lstdefinelanguage{JavaScript}{
  keywords={typeof, new, true, false, catch, function, return, null, catch, switch, var, if, in, while, do, else, case, break},
  keywordstyle=\color{blue}\bfseries,
  ndkeywords={class, export, boolean, throw, implements, import, this},
  ndkeywordstyle=\color{darkgray}\bfseries,
  identifierstyle=\color{black},
  sensitive=false,
  comment=[l]{//},
  morecomment=[s]{/*}{*/},
  commentstyle=\color{purple}\ttfamily,
  stringstyle=\color{red}\ttfamily,
  morestring=[b]',
  morestring=[b]"
}

% here you should include other packages with \usepackage

    \ifpdf

      % hyperref should be the last package loaded:
	  %\usepackage[pdftex]{hyperref}
      \usepackage{pst-pdf}
    \else

      % make the command \href from hyperref available as a 'print only'
      \newcommand{\href}[2]{#2}

    \fi

%Global Background must be put in preamble
\usebackgroundtemplate%
{%
    \includegraphics[width=\paperwidth,height=\paperheight]{Figuras/fundo.png}%
}
\setbeamertemplate{frametitle}[default][center]
 
\author{Othon Oliveira}
\title{Instrodução ao React.JS}
\institute{SENAC - PROA} 
\date{} 
%\subject{} 
\usepackage{dirtree}

\begin{document}

\begin{frame}
\titlepage
%\date{}
\end{frame}

% Capa - requer o TikZ
\newcommand{\capa}{
    \begin{tikzpicture}[remember picture,overlay]
        \node at (current page.south west)
            {\begin{tikzpicture}[remember picture, overlay]
                \fill[shading=radial,top color=orange,bottom color=orange,middle color=yellow] (0,0) rectangle (\paperwidth,\paperheight);
            \end{tikzpicture}
          };
    \end{tikzpicture}
}

\begin{frame}\frametitle{Sumário}
\tableofcontents
\end{frame}


%+++++++++++++++++++++++++++++++++++++++++++++++
\section{Estrutura de pastas dos nossos projetos}
%+++++++++++++++++++++++++++++++++++++++++++++++
\begin{frame}{Criando componentes}
\framesubtitle{Nossa pasta de projetos}
\begin{itemize}
	\item[•] Vamos criar nossos Componentes React numa pasta chamada componentes, que fica no \texttt{src}
	\item[•] Formato usual tipo ``CamelCase'' exemplo: \textbf{P}rimeiro\textbf{C}omponente.js
	\item[•] Nesse arquivo criamos um função, esta vai conter o código desse componente (a lógica)
	\item[•] Também precisamos ``exportar'' esta função, para reutilizá-la
	\item[•] Bora praticar!!
\end{itemize}
\end{frame}


%\begin{frame}{Componentes React}
%\documentclass{beamer}



\begin{frame}
\frametitle{Minha Imagem}
\begin{figure}
\centering
\includegraphics[width=0.5\textwidth]{Figuras/react1.png}
\caption{Componentes React}
\end{figure}
\end{frame}

%+++++++++++++++++++++++++++++++++++++++++++++++
\section{Criando um componente (do zero)}
%+++++++++++++++++++++++++++++++++++++++++++++++
\begin{frame}{Prmeiros passos}
\framesubtitle{Primeiros passo!!}
\begin{itemize}
	\item[•] Primeiro passo: criamos uma pasta para o projeto (no seu S.O.) ou no VSCode
	\pause
	\item[•] Segundo passo: \$ npx create-react-app seu-app
	\pause
	\item[•] Terceiro passo: entrar na pasta ``seu-app'' (cd seu-app)
	\pause
	\item[•] Quarto passo: inicia a aplicação: npm start
	\pause
	\item[•] Quinto passo: agora você cria a pasta componentes
\end{itemize}
\end{frame}




\begin{frame}
\frametitle{Estrutura de Diretórios do Projeto - Exemplo}
\dirtree{%
.1 seu-app/.
.2 README.md.
.2 node\_modules/.
.2 package.json.
.2 public/.
.3 index.html.
.3 favicon.ico.
.2 src/.
.3 App.js.
.3 index.js.
.3 logo.svg.
.3 componente/.
.4 MeuComponente.js.
.2 .gitignore.
}
\end{frame}
%------------------------------------------
\begin{frame}{Partes de um componente}
\framesubtitle{Terceira parte}

 \textbf{Acompanhe também pelo aquivo que o prof vai disponibilizar}

\end{frame}



\begin{frame}[fragile]{Exemplo}
\frametitle{MeuComponente.js}
\begin{lstlisting}[language=JavaScript]
import React from 'react';

function MeuComponente() {
  return (
    <div>
      Meu primeiro componente
    </div>
  );
}

export default MeuComponente;
\end{lstlisting}
\end{frame}


%------------ funçoes seta - slide 2 -------------------


\begin{frame}[fragile]{Exemplo com funções ``Seta''}
\frametitle{outra forma de: MeuComponente.js}
Isso está na mesma estrutua de pastas
\begin{lstlisting}[language=JavaScript]
const MeuComponente = () => {
    return(
        <div>
            <h1>Meu primeiro componente</h1>
        </div>
    );
};

export default MeuComponente;
\end{lstlisting}
\end{frame}




\begin{frame}{Utilizando o componente criado}
\framesubtitle{Importando é importante}
\begin{itemize}
	\item[•] A importação é a maneira que temos de reutilizar um componente
	\pause
	\item[•] Utilizamos a sintaxe: import X from './componente/X, onde ``X'' é o nome do componente
	\pause
	\item[•] Para colocar o componente importado em outro componente precisamos colocar na forma de tag: $<MeuComponente/>$
	\pause
	\item[•] Após esse passo finalizamos a importação do componente.

\end{itemize}
\end{frame}



\begin{frame}[fragile]
\frametitle{Usando um coponente criado}
\framesubtitle{No arquivo App.js}
\begin{lstlisting}[language=JavaScript]
//componentes
import MeuComponente from './componentes/MeuComponente';

// styles /css
import './App.css';

function App() {
  return (
    <div className="App"> // <-- className ??
      <h1>Fundamentos React</h1>
      <MeuComponente/> // <-- aqui vai nosso componente novo
    </div>
  );
}
export default App;
\end{lstlisting}
\end{frame}

%+++++++++++++++++++++++++++++++++++++++++++++++
\section{JSX}
%+++++++++++++++++++++++++++++++++++++++++++++++

\begin{frame}{JSX}
\framesubtitle{Algumas diferenças do HTML e Javascript original}
\begin{itemize}
	\item[•] JSX é o HTML do React
	\pause
	\item[•] É onde vamos declarar as \textbf{tags} que serão exibidas no navegador
	\pause
	\item[•] Ficam dentro do $return {  ... }$ do componente
	\pause
	\item[•] As instruções são semelhantes ao HTML e ao JS
	\pause
	\item[•] Temos algums diferenças do HTML, por exemplo: class agora é \textbf{className}
	\item[•] O JSX só pode ter \textbf{um elemento pai}
	
\end{itemize}
\end{frame}


%
\begin{frame}[fragile]
\frametitle{Exemplos de comentários}
\framesubtitle{Comentários no JSX}

\begin{lstlisting}[language=JavaScript]
const UmaFuncao = () => {
 // Um comentario de uma linha
 
 /*
   Um comentario de varias linhas
 */
return (
      <div>
       {/* Outro comentario */}
        <h1>Meu React</h1>
        <p className="Teste">Um texto</p>
      </div>
  );
};
\end{lstlisting}
\end{frame}

%+++++++++++++++++++++++++++++++++++++++++++++++
\section{Template Expressions}
%+++++++++++++++++++++++++++++++++++++++++++++++
\begin{frame}{JSX}
\framesubtitle{Javascript ``original'' e JSX, trocando figurinhas}
\begin{itemize}
	\item[•] Template Expression é o recurso que permite executar JS e JSX, interpolando variáveis
	\pause
	\item[•] Isso é muito útil para seus projetos React
	\pause
	\item[•] A sintaxe e $\{Algum codigo em JS \}$ do componente
	\pause
	\item[•] Tudo que está entre as chaves ${..}$ é processado em JS e retorna um resultado
	\pause
	\item[•] Temos algums diferenças do HTML, por exemplo: class agora é \textbf{className}
	\item[•] Vamos ver na prática
\end{itemize}
\end{frame}



\begin{frame}
\frametitle{Estrutura de Diretórios do Projeto - Exemplo}
\dirtree{%
.1 seu-app/.
.2 README.md.
.2 node\_modules/.
.2 package.json.
.2 public/.
.3 index.html.
.3 favicon.ico.
.2 src/.
.3 App.js.
.3 index.js.
.3 logo.svg.
.3 componente/.
.4 MeuComponente.js.
.4 TemplateExpression.js.
.2 .gitignore.
}
\end{frame}

%+++++++++++++++++++++++++++++++++++++++++++++++
\section{Hooks}
%+++++++++++++++++++++++++++++++++++++++++++++++

\begin{frame}[fragile]
\frametitle{Arrow Function}
\framesubtitle{Funções ``seta''}
Vamos criar esse componente??
\begin{verbatim}
 const dobrar = (num) => num * 2;
 console.log(dobrar(5)); // Isso imprimirá 10
\end{verbatim}
\pause
\begin{exampleblock}{Neste caso}
 A função dobrar aceita um argumento ``num'' e retorna o dobro desse número de forma concisa.
\end{exampleblock}
\pause
\begin{block}{Em resumo}
As funções de seta são uma adição útil ao JavaScript, especialmente para funções curtas e simples. Elas são particularmente úteis quando você deseja preservar o contexto do this do escopo circundante e tornar seu código mais legível e conciso em muitos casos.

\end{block}
\end{frame}
%-------------------------------------------------------------
\begin{frame}[fragile]{Para praticar}
\begin{exampleblock}{Create React App}
Certifique-se de ter um projeto React configurado
\end{exampleblock}
\begin{verbatim}
// slide 1/3
import React, { useState } from 'react';
function Calculator() {
  const [inputValue, setInputValue] = useState(''); // Estado para armazenar o valor inserido
  const [result, setResult] = useState(''); // Estado para
  armazenar o resultado
  // Função para lidar com a alteração do valor de entrada
  const handleInputChange = (event) => {
    setInputValue(event.target.value);
  };
// continua no proximo slide
  // Função para dobrar o valor inserido
  const doubleValue = () => {
    const value = parseFloat(inputValue); // Converte a entrada para um número
    if (!isNaN(value)) {
      setResult(`O dobro de ${value} é ${value * 2}`);
    } else {
      setResult('Por favor, insira um número válido.');
    }
  };

\end{verbatim}
//continua no próximo slide
\end{frame}




\begin{frame}[fragile]{Para praticar}
\begin{exampleblock}{Create React App}
Certifique-se de ter um projeto React configurado
\end{exampleblock}
\begin{verbatim}
// slide 2/3

  // Função para dobrar o valor inserido
  const doubleValue = () => {
    const value = parseFloat(inputValue); // Converte a entrada para um número
    if (!isNaN(value)) {
      setResult(`O dobro de ${value} é ${value * 2}`);
    } else {
      setResult('Por favor, insira um número válido.');
    }
  };
// continua no proximo slide
\end{verbatim}
//continua no próximo slide
\end{frame}



\begin{frame}[fragile]{Para praticar}
\begin{verbatim}
// slide 3/3
  return (
    <div>
      <h1>Calculadora Simples</h1>
      <input
        type="text"
        placeholder="Insira um valor"
        value={inputValue}
        onChange={handleInputChange}
      />
      <button onClick={doubleValue}>Dobrar</button>
      <p>{result}</p>
    </div>
  ); }
export default Calculator;
\end{verbatim}
\end{frame}



%-------------------------------------------------------------
\begin{frame}[fragile]
\frametitle{Importar o componente}
Agora é só importar o componente `Calculator'' para App.js

\begin{verbatim}
import React from 'react';
import Calculator from './Calculator';

function App() {
  return (
    <div className="App">
      <Calculator />
    </div>
  );
}
export default App;
\end{verbatim}

\end{frame}

%--------- para rodar ------------------------

\begin{frame}[fragile]
\frametitle{E para rodar ??}
\begin{block}{Passo 1}
Certifique-se de que seu servidor de desenvolvimento React esteja em execução (usando npm start ou algo semelhante).
\end{block}
\pause
\begin{block}{Passo 2}
Abra o navegador e acesse a página da aplicação. Você verá a interface da calculadora com um campo de entrada, um botão "Dobrar" e um espaço para exibir o resultado.
\end{block}
\pause
\begin{block}{Passo 3}
Insira um número no campo de entrada e clique no botão "Dobrar". O aplicativo calculará o dobro do valor inserido e o exibirá.
\end{block}
\end{frame}

%+++++++++++++++++++++++++++++++++++++++++++++++
\section{JSX}
%+++++++++++++++++++++++++++++++++++++++++++++++
%---------- Slide 1 ----------------------
\begin{frame}{Introdução ao JSX}
  \begin{block}{O que é JSX?}
    JSX (JavaScript XML) é uma extensão da sintaxe do JavaScript que permite você escrever código HTML/XML dentro do código JavaScript.
  \end{block}
%---------- Slide 2 ----------------------  
  \begin{block}{Por que usar JSX?}
    - Facilita a criação de interfaces de usuário.
    - Permite o uso de componentes React.
    - Pode ser transformado em código JavaScript puro pelo Babel.
    - Melhora a legibilidade do código.
  \end{block}
%---------- Slide 3 ----------------------  
    \begin{block}{Sintaxe Básica}
    - Use chaves `{}` para incorporar expressões JavaScript.
    - Use `<elemento>` para criar elementos JSX.
    - Os elementos JSX podem ter atributos e valores entre aspas.
  \end{block}
\end{frame}


\begin{frame}[fragile]
\frametitle{Um exemplo prático de JSX}
O JSX permite incorporar valores de variáveis, criar elementos HTML como $<h1>$ e $<p>$, e também criar listas como $<ul>$ e $<li>$. Ele facilita a construção de interfaces de usuário em React de forma legível e expressiva.

\begin{verbatim}
import React from 'react';
function App() {
  const name = 'React.js';
  return (
    <div>
      <h1>Exemplo de JSX</h1>
      <p>Bem-vindo ao {name}</p>
      <ul>
        <li>Componente 1</li>
        <li>Componente 2</li>
      </ul>
    </div>
  ); } export default App;
\end{verbatim}
\end{frame}





%------ slide 4 ------------------------------
\begin{frame}[fragile]{Passo a passo para criar um componente}
  \begin{block}{Passo 1 Desenvolvimento dos componentes}    
    \begin{itemize}
		\item[1] \textbf{Desenvolvimento local} Desenvolva os dois componentes em um ambiente local. Você pode criar os componentes em diretórios separados e desenvolvê-los usando um servidor de desenvolvimento local.	

		\item[2]\textbf{Teste e comportamento} Certifique-se de que os componentes funcionem conforme o esperado e tenham o comportamento desejado em um ambiente de desenvolvimento local.
	\end{itemize}
  \end{block}

  \begin{block}{Passo 2: Configuração do Ambiente de Produção}    
    \begin{itemize}
		\item[3] Certifique-se de que os componentes funcionem conforme o esperado e tenham o comportamento desejado em um ambiente de desenvolvimento local.
	\end{itemize}
  \end{block}

\end{frame}

%------ slide 5 ------------------------------
\begin{frame}[fragile]{Passo a passo para criar um componente}

  \begin{block}{Passo 3: Configuração do Ambiente de Produção}    
    \begin{itemize}
		\item[1] Certifique-se de que os componentes funcionem conforme o esperado e tenham o comportamento desejado em um ambiente de desenvolvimento local.
	\end{itemize}
  \end{block}

  \begin{block}{Passo 4: Criação de um Aplicativo de Exemplo}    
    \begin{itemize}
		\item[1]  Crie um aplicativo de exemplo que importa e usa os dois componentes. Isso permitirá que você teste os componentes em um contexto de aplicação..
	\end{itemize}
  \end{block}

  \begin{block}{Passo 5: Implantação dos Componentes}    
    \begin{itemize}
		\item[1] Implante os componentes em um servidor ou serviço de hospedagem. Você pode usar uma variedade de opções, como Netlify, Vercel, GitHub Pages, ou implantar em seu próprio servidor.
	\end{itemize}
  \end{block}

\end{frame}


%------ slide 6 ------------------------------
\begin{frame}[fragile]{Passo a passo para criar um componente}

  \begin{block}{Passo 6: Publicação dos Componentes}    
    \begin{itemize}
		\item[1] Disponibilize os componentes para uso público. Isso pode envolver a criação de um pacote npm privado, publicando-os em um repositório de pacotes privado ou até mesmo como arquivos estáticos no servidor.
	\end{itemize}
  \end{block}

  \begin{block}{Passo 7: Integração com Outros Projetos}    
    \begin{itemize}
		\item[1] Em outros projetos onde você deseja usar esses componentes, você pode instalá-los como pacotes npm ou incluir os arquivos diretamente em seu código, dependendo de como você os implantou.
	\end{itemize}
  \end{block}

  \begin{block}{Passo 8: Monitoramento e Manutenção}    
    \begin{itemize}
		\item[1] Monitore o desempenho e a funcionalidade dos componentes em produção. Esteja pronto para realizar correções e atualizações conforme necessário.
	\end{itemize}
  \end{block}

\end{frame}

%%+++++++++++++++++++++++++++++++++++++++++++++++
\begin{frame}[fragile]
\frametitle{Um exemplo prático de JSX}
Dentro do diretório components,
crie um arquivo chamado Componente1.js:

\begin{verbatim}
// touch components/Componente1.js (um editor de texto qq)
import React from 'react';

function Componente1() {
  return (
    <div>
      <h2>Componente 1</h2>
      <p>Este é o Componente 1. 
      Clique em "Componente 1" para vê-lo em ação.</p>
    </div>
  );
}
export default Componente1;
\end{verbatim}
\end{frame}

%+++++++++++++++++++++++++++++++++++++++++++++++
\begin{frame}[fragile]
\frametitle{Outro exemplo prático de JSX}
Ajuste na 1ª função (que tem componente1..)

\begin{verbatim}
  return (
    <div>
      <h1>Exemplo de Clique em Componente</h1>
      <ul>
        <li
          onClick={() => handleComponentClick('Componente 1')}
          className={activeComponent === 'Componente 1' ? 'active' : ''}
        >
          Componente 1
        </li>
        <li
          onClick={() => handleComponentClick('Componente 2')}
          className={activeComponent === 'Componente 2' ? 'active' : ''}
        > continua abaixo ....
\end{verbatim}
\end{frame}



\begin{frame}[fragile]
\frametitle{Um exemplo prático de JSX}
Ajuste na 1ª função (que tem componente2..)

\begin{verbatim}
      .. contunuação ...
          Componente 2
        </li>
      </ul>
      {activeComponent === 'Componente 1' && (
        <div>
          <h2>Componente 1</h2>
          <p>Este é o Componente 1. Clique em "Componente 1" para vê-lo em ação.</p>
        </div>
      )}
      {activeComponent === 'Componente 2' && (
        <div>
          <h2>Componente 2</h2>
          <p>Este é o Componente 2. Clique em "Componente 2" para vê-lo em ação.</p>
        </div>
      )}
    </div> ); }  export default App;
\end{verbatim}
\end{frame}

%-----------------------------------------------

\end{document}