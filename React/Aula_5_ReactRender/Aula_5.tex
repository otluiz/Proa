%++++++% Preâmbulo %+++++++++++++++++++++++++++++++++++++++++++++++++++++++++
\documentclass[13pt, xcolor={dvipsnames,svgnames}, portuguese]{beamer}
%\documentclass[[11pt, xcolor={dvipsnames,svgnames,table},portuguese]{beamer} 

\usetheme{CambridgeUS}

\setbeamercolor*{structure}{bg=PineGreen!20,fg=PineGreen} %fg=PineGreen
\definecolor{beamer@pinegreen}{rgb}{0.137,0.666,0.741}


\setbeamercolor*{palette primary}{use=structure,fg=white,bg=structure.fg}
\setbeamercolor*{palette secondary}{use=structure,fg=white,bg=structure.fg!75}
\setbeamercolor*{palette tertiary}{use=structure,fg=white,bg=structure.fg!50!black}
\setbeamercolor*{palette quaternary}{fg=white,bg=black}

\setbeamercolor{section in toc}{fg=black,bg=white}
\setbeamercolor{alerted text}{use=structure,fg=structure.fg!50!black!80!black}

\setbeamercolor{titlelike}{parent=palette primary,fg=structure.fg!50!black}
\setbeamercolor{frametitle}{bg=gray!10!white,fg=PineGreen}

\setbeamercolor*{titlelike}{parent=palette primary}

\usepackage[utf8]{inputenc}
\usepackage[brazil]{babel}  % idioma
\usepackage{amsmath,amsfonts,amssymb,textcomp}
\usepackage{graphicx}
\usepackage{subfigure}
\usepackage[utf8]{inputenc}
\usepackage{ifpdf}
\usepackage{listings}

% Configurações para o ambiente lstlisting
\lstset{
    language=C,
    basicstyle=\ttfamily\footnotesize,
    numbers=left,
    numberstyle=\tiny,
    numbersep=5pt
}

\lstdefinelanguage{JavaScript}{
  keywords={typeof, new, true, false, catch, function, return, null, catch, switch, var, if, in, while, do, else, case, break},
  keywordstyle=\color{blue}\bfseries,
  ndkeywords={class, export, boolean, throw, implements, import, this},
  ndkeywordstyle=\color{darkgray}\bfseries,
  identifierstyle=\color{black},
  sensitive=false,
  comment=[l]{//},
  morecomment=[s]{/*}{*/},
  commentstyle=\color{purple}\ttfamily,
  stringstyle=\color{red}\ttfamily,
  morestring=[b]',
  morestring=[b]"
}


% here you should include other packages with \usepackage

    \ifpdf

      % hyperref should be the last package loaded:
	  %\usepackage[pdftex]{hyperref}
      \usepackage{pst-pdf}
    \else

      % make the command \href from hyperref available as a 'print only'
      \newcommand{\href}[2]{#2}

    \fi

%Global Background must be put in preamble
\usebackgroundtemplate%
{%
    \includegraphics[width=\paperwidth,height=\paperheight]{Figuras/fundo.png}%
}
\setbeamertemplate{frametitle}[default][center]
 
\author{Othon Oliveira}
\title{Introdução ao React.JS}
\institute{SENAC - PROA} 
\date{} 
%\subject{} 

\begin{document}

\begin{frame}
\titlepage
%\date{}
\end{frame}

% Capa - requer o TikZ
\newcommand{\capa}{
    \begin{tikzpicture}[remember picture,overlay]
        \node at (current page.south west)
            {\begin{tikzpicture}[remember picture, overlay]
                \fill[shading=radial,top color=orange,bottom color=orange,middle color=yellow] (0,0) rectangle (\paperwidth,\paperheight);
            \end{tikzpicture}
          };
    \end{tikzpicture}
}

\begin{frame}\frametitle{Sumário}
\tableofcontents
\end{frame}


%+++++++++++++++++++++++++++++++++++++++++++++++
\section{Entrada de dados}
%+++++++++++++++++++++++++++++++++++++++++++++++
\begin{frame}[fragile]
\frametitle{Entrada de Dados em Componentes React}
\framesubtitle{[``algumaCoisa'', ``setAlgumaCoisa'']}
\begin{itemize}
    \item Para permitir que os usuários insiram dados em um componente React, geralmente usamos elementos de entrada, como \texttt{<input type="text" />} para entrada de texto.
    \item Em um componente React, podemos usar o estado para armazenar o valor do input e a função \texttt{useState} para atualizá-lo.
\end{itemize}
Exemplo:
    \begin{lstlisting}[language=JavaScript]
    function InputComponent() {
        const [inputValue, setInputValue] = useState('');
        return (
            <div>
                <label>Digite algo:</label>
                <input 
                    type="text" 
                    value={inputValue} 
                    onChange={(e) => setInputValue(e.target.value)} 
                /> <p>Voce digitou: {inputValue}</p>
            </div>
        ); }
    \end{lstlisting}

\end{frame}

%+++++++++++++++++++++++++++++++++++++++++++++++
\section{Atualizando o Input}
%+++++++++++++++++++++++++++++++++++++++++++++++

\begin{frame}{Atualizando o Estado com a Entrada do Usuário}
\begin{itemize}
    \item No exemplo acima, o valor do input é armazenado no estado \texttt{inputValue} e atualizado usando \texttt{setInputValue}.
    \item O evento \texttt{onChange} é usado para detectar mudanças no input. Quando o usuário digita algo, a função passada para \texttt{onChange} é chamada, atualizando o estado com o novo valor do input.
    \item Dessa forma, o componente React pode reagir dinamicamente às entradas do usuário.
\end{itemize}
\end{frame}

%-------------------------------------------------------------
\begin{frame}[fragile]
\frametitle{Exercício anterior, outro exemplo}
\framesubtitle{Uma possível solução}
 \begin{lstlisting}[language=JavaScript]
    const SomaComponente = () => {
	 const valor1 = 10; // Primeiro valor constante
	 const valor2 = 20; // Segundo valor constante
	 const [resultado, setResultado] = useState(null);
	 const calcularSoma = () => {
	 const soma = valor1 + valor2;
		setResultado(soma);
	 };
	 return (
		<div>
		<h1>Calculadora de Soma</h1>
		<button onClick={calcularSoma}>Calcular Soma</button>
		{resultado !== null && <p>{valor1} + {valor2} =
		{resultado}</p>}
	 	</div>
	 );
	}
    \end{lstlisting}

\end{frame}


%+++++++++++++++++++++++++++++++++++++++++++++++
\section{Calculos com }
%+++++++++++++++++++++++++++++++++++++++++++++++
\begin{frame}{Atualizando o Estado com a Entrada do Usuário}

Agora melhore o componente fazendo 3 (três) novos botões que calculam a subtração a multiplicação e a divisão, os inputs devem ser os mesmos.
\end{frame}





\begin{frame}{Atualizando o Estado com a Entrada do Usuário}
 Crie uma função que recebe o valor do usuário, em um “input text” e
retorna um valor segundo a função a seguir
Exemplo:\\
const dobrar $= (num) => num * 2;$\\
console.log(dobrar(5)); // Isso imprimirá 10
\end{frame}


%-------------------------------------------------------------
%----------------------------


%+++++++++++++++++++++++++++++++++++++++++++++++
\section{Calculos com }
%+++++++++++++++++++++++++++++++++++++++++++++++

%-------------------------------------------------------------

%-------------------------------------------------------------


%%+++++++++++++++++++++++++++++++++++++++++++++++


%+++++++++++++++++++++++++++++++++++++++++++++++




%-----------------------------------------------

\end{document}
