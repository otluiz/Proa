%++++++% Preâmbulo %+++++++++++++++++++++++++++++++++++++++++++++++++++++++++
\documentclass[13pt, xcolor={dvipsnames,svgnames}, portuguese]{beamer}
%\documentclass[[11pt, xcolor={dvipsnames,svgnames,table},portuguese]{beamer} 

\usetheme{CambridgeUS}

\setbeamercolor*{structure}{bg=PineGreen!20,fg=PineGreen} %fg=PineGreen
\definecolor{beamer@pinegreen}{rgb}{0.137,0.666,0.741}
\setbeamercolor*{palette primary}{use=structure,fg=white,bg=structure.fg}
\setbeamercolor*{palette secondary}{use=structure,fg=white,bg=structure.fg!75}
\setbeamercolor*{palette tertiary}{use=structure,fg=white,bg=structure.fg!50!black}
\setbeamercolor*{palette quaternary}{fg=white,bg=black}

\setbeamercolor{section in toc}{fg=black,bg=white}
\setbeamercolor{alerted text}{use=structure,fg=structure.fg!50!black!80!black}

\setbeamercolor{titlelike}{parent=palette primary,fg=structure.fg!50!black}
\setbeamercolor{frametitle}{bg=gray!10!white,fg=PineGreen}

\setbeamercolor*{titlelike}{parent=palette primary}

\usepackage[utf8]{inputenc}
\usepackage[brazil]{babel}  % idioma
\usepackage{amsmath,amsfonts,amssymb,textcomp}
\usepackage{graphicx}
\usepackage{subfigure}
\usepackage[utf8]{inputenc}
\usepackage{ifpdf}
\usepackage{listings}
\usepackage{dirtree}

% Configurações para o ambiente lstlisting
\lstset{
    language=C,
    basicstyle=\ttfamily\footnotesize,
    numbers=left,
    numberstyle=\tiny,
    numbersep=5pt
}
\lstdefinelanguage{JavaScript}{
  keywords={typeof, new, true, false, catch, function, return, null, catch, switch, var, if, in, while, do, else, case, break},
  keywordstyle=\color{blue}\bfseries,
  ndkeywords={class, export, boolean, throw, implements, import, this},
  ndkeywordstyle=\color{darkgray}\bfseries,
  identifierstyle=\color{black},
  sensitive=false,
  comment=[l]{//},
  morecomment=[s]{/*}{*/},
  commentstyle=\color{purple}\ttfamily,
  stringstyle=\color{red}\ttfamily,
  morestring=[b]',
  morestring=[b]"
}
% here you should include other packages with \usepackage

    \ifpdf

      % hyperref should be the last package loaded:
	  %\usepackage[pdftex]{hyperref}
      \usepackage{pst-pdf}
    \else

      % make the command \href from hyperref available as a 'print only'
      \newcommand{\href}[2]{#2}

    \fi

%Global Background must be put in preamble
\usebackgroundtemplate%
{%
    \includegraphics[width=\paperwidth,height=\paperheight]{Figuras/fundo.png}%
}
\setbeamertemplate{frametitle}[default][center]
 
\author{Othon Oliveira}
\title{Instrodução ao React.JS}
\institute{SENAC - PROA} 
\date{} 
%\subject{} 

\begin{document}

\begin{frame}
\titlepage
%\date{}
\end{frame}

% Capa - requer o TikZ
\newcommand{\capa}{
    \begin{tikzpicture}[remember picture,overlay]
        \node at (current page.south west)
            {\begin{tikzpicture}[remember picture, overlay]
                \fill[shading=radial,top color=orange,bottom color=orange,middle color=yellow] (0,0) rectangle (\paperwidth,\paperheight);
            \end{tikzpicture}
          };
    \end{tikzpicture}
}

%+++++++++++++++++++++++ Telas iniciais  ---------------------------------
\begin{frame}\frametitle{Sumário}
\tableofcontents
\end{frame}
%---------------------------------------------------------------------
\begin{frame}
\frametitle{Minha Imagem}
\begin{figure}
\centering
\includegraphics[width=0.5\textwidth]{Figuras/react1.png}
\caption{Componentes React}
\end{figure}
\end{frame}

%+++++++++++++++++++++++++++++++++++++++++++++++++++++++++++++++++++++
\section{Ract Router}
%+++++++++++++++++++++++++++++++++++++++++++++++
\begin{frame}{O que é React Router}
\framesubtitle{Um dos pacotes mais utilizados}
	\begin{block}{Criando estruturas de rotas para cada página do nosso app}
		\begin{itemize}
		  \item[a.] É um pacote externo. É preciso instalar esse pacote
		  \pause
		  \item[b.] comando: \textbf{npm install react-router-dom}
		   \pause		  
		  \item[c.] Multiplas funções como Redirect, Nested Routes, Not Found Routes
		  \pause
		  \item[d.] E muitas outras funcionalidades
		\end{itemize}
	\end{block} 
\end{frame}


\begin{frame}[fragile]{Exemplo}
\frametitle{App.js}
\begin{lstlisting}[language=JavaScript]
<Route path="/" element={LoginForm} />
<Route path="/pagina-inicial" element={PaginaInicial} />
\end{lstlisting}

\begin{alertblock}{Definindo rotas}
 $path="/"$ Raiz da aplicação \\
 $path="/pagina-inicial"$ Página inicial da aplicação
\end{alertblock}

\begin{alertblock}{Definindo rotas}
 Bora praticar?
\end{alertblock}

\end{frame}



\begin{frame}
\frametitle{Estrutura de Diretórios do Projeto - Exemplo}
\dirtree{%
.1 seu-app/.
.2 README.md.
.2 node\_modules/.
.2 package.json.
.2 public/.
.3 index.html.
.3 favicon.ico.
.2 src/.
.3 App.js.
.3 index.js.
.3 logo.svg.
.3 paginas/.
.4 Home.js.
.4 Sobre.js.
.4 Home.css.
.2 .gitignore.
}
\end{frame}


\begin{frame}[fragile]
\frametitle{Exemplo}
\framesubtitle{App.js}
\begin{lstlisting}[language=JavaScript]
import App.js
// configurar o react browser route
import { BrowserRouter as Router, Route, Switch } from 'react-router-dom';
//paginas
import {Home} from './paginas/Home';
import {Sobre} from './paginas/Sobre';

const App = () => {
  return (
    <Router>
      <div className="App">
        <Routes>
		  <Route path="/" element={<Home />} />
		  <Route path="/sobre" element={<Sobre />} />
        </Routes>
      </div>
    </Router>
  );
};
\end{lstlisting}

\end{frame}


\begin{frame}{O que é React Router}
\framesubtitle{Criar a estrutura de diretórios acima}
	\begin{block}{Criando estruturas de rotas para cada página do nosso app}
		\begin{itemize}
		  \item[a.] \textbf{Nested Routes :} O redirecionamento permite mudar o conteúdo de uma página quando o usuário interage com ela. 
		  \pause
		  \item[b.] Você pode, por exemplo, redirecionar de uma antiga página para uma nova, ou para o inicio da aplicação
		   \pause		  
		   
		   <Route path="/" exact component={LoginForm} />
		  \item[c.]  
		  \pause
		  \item[d.]
		\end{itemize}
	\end{block} 
\end{frame}
%+++++++++++++++++++++++++++++++++++++++++++++++
\section{Projeto Instagrão}
%+++++++++++++++++++++++++++++++++++++++++++++++

%-------------------------------------------------------------


%-------------------------------------------------------------



%-------------------------------------------------------------

%-------------------------------------------------------------


%%+++++++++++++++++++++++++++++++++++++++++++++++

\end{document}

