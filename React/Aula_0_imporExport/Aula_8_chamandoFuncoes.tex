
%++++++% Preâmbulo %+++++++++++++++++++++++++++++++++++++++++++++++++++++++++
\documentclass[13pt, xcolor={dvipsnames,svgnames}, portuguese]{beamer}
%\documentclass[[11pt, xcolor={dvipsnames,svgnames,table},portuguese]{beamer} 

\usetheme{CambridgeUS}

\setbeamercolor*{structure}{bg=PineGreen!20,fg=PineGreen} %fg=PineGreen
\definecolor{beamer@pinegreen}{rgb}{0.137,0.666,0.741}


\setbeamercolor*{palette primary}{use=structure,fg=white,bg=structure.fg}
\setbeamercolor*{palette secondary}{use=structure,fg=white,bg=structure.fg!75}
\setbeamercolor*{palette tertiary}{use=structure,fg=white,bg=structure.fg!50!black}
\setbeamercolor*{palette quaternary}{fg=white,bg=black}

\setbeamercolor{section in toc}{fg=black,bg=white}
\setbeamercolor{alerted text}{use=structure,fg=structure.fg!50!black!80!black}

\setbeamercolor{titlelike}{parent=palette primary,fg=structure.fg!50!black}
\setbeamercolor{frametitle}{bg=gray!10!white,fg=PineGreen}

\setbeamercolor*{titlelike}{parent=palette primary}

\usepackage[utf8]{inputenc}
\usepackage[brazil]{babel}  % idioma
\usepackage{amsmath,amsfonts,amssymb,textcomp}
\usepackage{graphicx}
\usepackage{subfigure}
\usepackage[utf8]{inputenc}
\usepackage{ifpdf}
\usepackage{listings}

% Configurações para o ambiente lstlisting
\lstset{
    language=C,
    basicstyle=\ttfamily\footnotesize,
    numbers=left,
    numberstyle=\tiny,
    numbersep=5pt
}

% here you should include other packages with \usepackage

    \ifpdf

      % hyperref should be the last package loaded:
	  %\usepackage[pdftex]{hyperref}
      \usepackage{pst-pdf}
    \else

      % make the command \href from hyperref available as a 'print only'
      \newcommand{\href}[2]{#2}

    \fi

%Global Background must be put in preamble
\usebackgroundtemplate%
{%
    \includegraphics[width=\paperwidth,height=\paperheight]{Figuras/fundo.png}%
}
\setbeamertemplate{frametitle}[default][center]
 
\author{Othon Oliveira}
\title{Introdução às funções externas Java Script}
\institute{SENAC - PROA} 
\date{} 
%\subject{} 

\begin{document}

\begin{frame}
\titlepage
%\date{}
\end{frame}

% Capa - requer o TikZ
\newcommand{\capa}{
    \begin{tikzpicture}[remember picture,overlay]
        \node at (current page.south west)
            {\begin{tikzpicture}[remember picture, overlay]
                \fill[shading=radial,top color=orange,bottom color=orange,middle color=yellow] (0,0) rectangle (\paperwidth,\paperheight);
            \end{tikzpicture}
          };
    \end{tikzpicture}
}

\begin{frame}\frametitle{Sumário}
\tableofcontents
\end{frame}


%+++++++++++++++++++++++++++++++++++++++++++++++
\section{Entrada de dados externo}
%+++++++++++++++++++++++++++++++++++++++++++++++
\begin{frame}
  \frametitle{O conceito de \textbf{Módulo} em JavaScript}
  Em JavaScript, um módulo é um arquivo que encapsula e organiza funcionalidades relacionadas. Ele pode conter variáveis, funções, classes e outros elementos. Através do \texttt{import}, você pode acessar esses elementos de outros arquivos.
\end{frame}

\begin{frame}[fragile]
  \frametitle{Exemplo 1: Importar uma função específica de um módulo}
  Suponha que você tenha um módulo \texttt{meuModulo.js}:
  \begin{verbatim}
    // meuModulo.js
    export function minhaFuncao() {
      console.log("Executando minha função!");
    }
  \end{verbatim}
  
  No seu arquivo principal:
  \begin{verbatim}
    // Arquivo principal
    import { minhaFuncao } from './meuModulo';
    minhaFuncao();
  \end{verbatim}
\end{frame}

\begin{frame}[fragile]
  \frametitle{Exemplo 2: Importar várias funcionalidades de um módulo}
  No módulo \texttt{meuModulo.js}:
  \begin{verbatim}
    // meuModulo.js
    export function funcao1() { /* ... */ }
    export function funcao2() { /* ... */ }
    export function funcao3() { /* ... */ }
  \end{verbatim}
  
  No seu arquivo principal:
  \begin{verbatim}
    // Arquivo principal
    import { funcao1, funcao2, funcao3 } from './meuModulo';
    funcao1();
    funcao2();
    funcao3();
  \end{verbatim}
\end{frame}

\begin{frame}[fragile]
  \frametitle{Exemplo 3: Importar todo o módulo para usar suas funcionalidades}
  No módulo \texttt{meuModulo.js}:
  \begin{verbatim}
    // meuModulo.js
    export function minhaFuncao() { /* ... */ }
    export function outraFuncao() { /* ... */ }
  \end{verbatim}
  
  No seu arquivo principal:
  \begin{verbatim}
    // Arquivo principal
    import * as meuModulo from './meuModulo';
    meuModulo.minhaFuncao();
    meuModulo.outraFuncao();
  \end{verbatim}
\end{frame}
%------------------------------------------------
\begin{frame}[fragile]
  \frametitle{Chamando Função JavaScript de um Arquivo Externo}

  Considere a estrutura de pastas:

  \begin{verbatim}
    projeto/
    |-- index.html
    |-- js/
        |-- funcoes.js
  
  \end{verbatim}

  No arquivo \texttt{funcoes.js} (dentro da pasta \texttt{js/}):

  \begin{verbatim}
    // funcoes.js
    function mostrarMensagem() {
      alert("Função chamada!");
    }
  \end{verbatim}
\end{frame}
%-------------------------------------------------------------
\begin{frame}[fragile]
  \frametitle{Chamando Função JavaScript de um Arquivo Externo}

  No arquivo \texttt{index.html}:

  \begin{verbatim}
    <!DOCTYPE html>
    <html lang="en">
    <head>
      <meta charset="UTF-8">
      <meta name="viewport" content="width=device-width, initial-scale=1.0">
      <title>Chamando Função JavaScript</title>
      <script src="js/funcoes.js"></script>
    </head>
    <body>
      <button onclick="mostrarMensagem()">Clique-me</button>
    </body>
    </html>
  \end{verbatim}
  Ao clicar no botão, a função \texttt{mostrarMensagem()} do arquivo \texttt{funcoes.js} será chamada.
\end{frame}
%----------------------------------------------------
\begin{frame}[fragile]
  \frametitle{Importação e Exportação de Funções}

  Considere a estrutura de pastas:

  \begin{verbatim}
    projeto/
    |-- index.html
    |-- js/
        |-- funcoes.js
        |-- modulo.js

  \end{verbatim}

  No arquivo \texttt{modulo.js} (dentro da pasta \texttt{js/}):

  \begin{verbatim}
    // modulo.js
    export function mostrarMensagem() {
      alert("Função chamada!");
    }
  \end{verbatim}
\end{frame}
%-----------------------------------------------------------
\begin{frame}[fragile]
  \frametitle{Exemplo: Importação e Exportação de Funções}

  No arquivo \texttt{funcoes.js} (dentro da pasta \texttt{js/}):

  \begin{verbatim}
    // funcoes.js
    import { mostrarMensagem } from './modulo.js';
    
    export function outraFuncao() {
      console.log("Outra função chamada!");
      mostrarMensagem();
    }
  \end{verbatim}
\end{frame}

\begin{frame}[fragile]
  \frametitle{Exemplo: Importação e Exportação de Funções}

  No arquivo \texttt{index.html}:

  \begin{verbatim}
    <!DOCTYPE html>
    <html lang="en">
    <head>
      <meta charset="UTF-8">\\
      
      <title>Chamando Função JavaScript</title>
      <script type="module" src="js/funcoes.js"></script>
    </head>
    <body>
      <button onclick="outraFuncao()">Clique-me</button>
    </body>
    </html>
  \end{verbatim}
  Ao clicar no botão, a função \texttt{outraFuncao()} do arquivo \texttt{funcoes.js} será chamada, e ela por sua vez chamará a função \texttt{mostrarMensagem()} do arquivo \texttt{modulo.js}.
\end{frame}
%----------------------------------------------------------------

%+++++++++++++++++++++++++++++++++++++++++++++++
\section{Desafio}
%+++++++++++++++++++++++++++++++++++++++++++++++

\begin{frame}[fragile]
\frametitle{Exercícios de Estruturas de Repetição}

\textbf{Exercício 1:} Imprima os números de 1 a 10.

\textbf{Exercício 2:} Imprima os números pares de 2 a 20.

\textbf{Exercício 3:} Calcule a soma dos números de 1 a 100.

\textbf{Exercício 4:} Imprima a tabuada do 5 (de 1 a 10).

\textbf{Exercício 5:} Conte quantas vogais há em uma palavra.

\textbf{Exercício 6:} Peça ao usuário um número e imprima os números de 1 até esse número.

\textbf{Exercício 7:} Peça ao usuário para digitar números até que ele digite zero. Imprima a soma dos números.

\textbf{Exercício 8:} Peça ao usuário para adivinhar um número entre 1 e 10. Continue pedindo até ele acertar.

\textbf{Exercício 9:} Peça ao usuário para digitar 5 números e imprima a média deles (utilize estrutura de repetição).

\textbf{Exercício 10:} Peça ao usuário para digitar um número e calcule seu fatorial.

\end{frame}
%-------------------------------------------------------------
\begin{frame}[fragile]
\frametitle{Exemplo de formulário para apresentação da solução}

\begin{verbatim}

<!DOCTYPE html>
<html>
<head>
  <title>Exercício 1</title>
  <script src="../scripts/script.js"></script>
</head>
<body>
  <h1>Exercício 1</h1>
  <button onclick="exercicio1()">Clique para Executar</button>
</body>
</html>
\end{verbatim}

\end{frame}

\begin{frame}[fragile]
\frametitle{Exemplo de formulário para apresentação da solução}

\begin{verbatim}

<!DOCTYPE html>
<html>
<head>
  <title>Exercício 1</title>
  <script src="./script.js"></script>
</head>
<body>
  <h1>Exercício 1</h1>
  <button onclick="exercicio1()">Clique para Executar</button>
</body>
</html>
\end{verbatim}

\end{frame}
%%+++++++++++++++++++++++++++++++++++++++++++++++

\begin{frame}
\frametitle{Resolva os exercícios que abordam o uso de funções e estrutura de repetição}

\begin{itemize}
	\item[1.] \textbf{Soma dos número pares:} Crie uma função que receba um número inteiro positivo como parâmetro e retorne a soma de todos os números pares de 1 até esse número
	\item[2.] \textbf{Sequência de Fibonacci:} Crie uma função que gere e retorne os primeiros n números da sequência de Fibonacci. A sequência começa com 0 e 1, e cada número subsequente é a soma dos dois números anteriores.
	\item[3.] \textbf{Contagem de dígitos:} Crie uma função que conte e retorne quantos dígitos um número inteiro possui. Por exemplo, o número 12345 possui 5 dígitos.
	\item[4.] \textbf{Soma de dígitos:} Crie uma função que some os dígitos de um número e verifique se ele é divisível por 3
\end{itemize}

\end{frame}
%%-------------------------------------------------------------
%\begin{frame}[fragile]
%\frametitle{Exemplo de Uso da Média Aritmética}
%Vamos calcular a média aritmética de um conjunto de valores.
%\begin{verbatim}
%Vamos calcular a média aritmética dos valores 10, 15, 20, 25 e 30.
%
%\begin{verbatim}
%const valor1 = 10, valor2 = 15, valor3 = 20;
%const valor4 = 25, valor5 = 30;
%
%const soma = valor1 + valor2 + valor3 + valor4 + valor5;
%const quantidade = 5;
%const media = soma / quantidade;
%
%console.log("Soma:", soma);
%console.log("Média Aritmética:", media);
%
%\end{verbatim}
%Pergunta: Onde deverei "colar" esse código, para que funcione
%do jeito que está?
%\end{frame}
%
%%+++++++++++++++++++++++++++++++++++++++++++++++
%\section{Calculos com mais um pouco de desafios}
%%+++++++++++++++++++++++++++++++++++++++++++++++
%\begin{frame}
%\frametitle{Apresentando a Média Ponderada}
%A média ponderada é uma medida de centralidade que leva em consideração diferentes pesos atribuídos a cada valor no cálculo da média.
%\vspace{0.5cm}
%Para calcular a média ponderada:
%
%\begin{enumerate}
%\item Multiplique cada valor pelo seu respectivo peso.
%\item Some os produtos obtidos.
%\item Divida a soma pelo total dos pesos.
%\end{enumerate}
%
%\vspace{0.5cm}
%
%Isso é expresso pela fórmula:
%\[
%\text{Média Ponderada} = \frac{v_1 \cdot p_1 + v_2 \cdot p_2 + v_3 \cdot p_3 + \ldots}{p_1 + p_2 + p_3 + \ldots}
%\]
%onde $v_i$ é cada valor individual do conjunto e $p_i$ é o peso correspondente a esse valor.
%
%\end{frame}
%-------------------------------------------------------------
%\begin{frame}[fragile]
%\frametitle{Exemplo de Cálculo de Juros Simples}
%
%Aqui está um exemplo de cálculo de juros simples em HTML:
%
%\begin{verbatim}
%<!DOCTYPE html>
%<html>
%<head> <title>Cálculo de Juros Simples</title> </head>
%<body>
%  <h1>Calculadora de Juros Simples</h1>
%  <script>
%    const principal = 5000;
%    const taxaDeJuros = 10; // 10%
%    const periodo = 2;
%    const montante = principal + (principal * taxaDeJuros / 100 * periodo);
%    const resultado = `Montante com Juros Simples: $${montante.toFixed(2)}`;
%    document.write(resultado);
%  </script>
%</body>
%</html>
%\end{verbatim}
%
%\end{frame}
%%-------------------------------------------------------------
%\begin{frame}[fragile]
%\frametitle{Exemplo de Cálculo de Juros Simples}
%Aqui está um exemplo de cálculo de juros simples em HTML com prompts de comando:
%
%\begin{verbatim}
%<!DOCTYPE html>
%<html><head>
%  <title>Cálculo de Juros Simples</title></head>
%<body>
%  <h1>Calculadora de Juros Simples</h1>
%  <script>
%    const principal = parseFloat(prompt("Digite o valor principal:"));
%    const taxaDeJuros = parseFloat(prompt("Digite a taxa de juros (%):"));
%    const periodo = parseFloat(prompt("Digite o período (anos):"));
%    const montante = principal + (principal * taxaDeJuros / 100 * periodo);
%    const resultado = `Montante com Juros Simples: $${montante.toFixed(2)}`;
%    document.write(resultado);
%  </script>
%</body>
%</html>
%\end{verbatim}
%
%\end{frame}

%%+++++++++++++++++++++++++++++++++++++++++++++++


%+++++++++++++++++++++++++++++++++++++++++++++++




%-----------------------------------------------

\end{document}