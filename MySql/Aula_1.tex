%++++++% Preâmbulo %+++++++++++++++++++++++++++++++++++++++++++++++++++++++++
\documentclass[13pt, xcolor={dvipsnames,svgnames}, portuguese]{beamer}
%\documentclass[[11pt, xcolor={dvipsnames,svgnames,table},portuguese]{beamer} 

\usetheme{CambridgeUS}

\setbeamercolor*{structure}{bg=PineGreen!20,fg=PineGreen} %fg=PineGreen
\definecolor{beamer@pinegreen}{rgb}{0.137,0.666,0.741}


\setbeamercolor*{palette primary}{use=structure,fg=white,bg=structure.fg}
\setbeamercolor*{palette secondary}{use=structure,fg=white,bg=structure.fg!75}
\setbeamercolor*{palette tertiary}{use=structure,fg=white,bg=structure.fg!50!black}
\setbeamercolor*{palette quaternary}{fg=white,bg=black}

\setbeamercolor{section in toc}{fg=black,bg=white}
\setbeamercolor{alerted text}{use=structure,fg=structure.fg!50!black!80!black}

\setbeamercolor{titlelike}{parent=palette primary,fg=structure.fg!50!black}
\setbeamercolor{frametitle}{bg=gray!10!white,fg=PineGreen}

\setbeamercolor*{titlelike}{parent=palette primary}

\usepackage[utf8]{inputenc}
\usepackage[brazil]{babel}  % idioma
\usepackage{amsmath,amsfonts,amssymb,textcomp}
\usepackage{graphicx}
\usepackage{subfigure}
\usepackage[utf8]{inputenc}
\usepackage{ifpdf}
\usepackage{listings}

% Configurações para o ambiente lstlisting
\lstset{
    language=C,
    basicstyle=\ttfamily\footnotesize,
    numbers=left,
    numberstyle=\tiny,
    numbersep=5pt
}
\lstdefinelanguage{JavaScript}{
  keywords={typeof, new, true, false, catch, function, return, null, catch, switch, var, if, in, while, do, else, case, break},
  keywordstyle=\color{blue}\bfseries,
  ndkeywords={class, export, boolean, throw, implements, import, this},
  ndkeywordstyle=\color{darkgray}\bfseries,
  identifierstyle=\color{black},
  sensitive=false,
  comment=[l]{//},
  morecomment=[s]{/*}{*/},
  commentstyle=\color{purple}\ttfamily,
  stringstyle=\color{red}\ttfamily,
  morestring=[b]',
  morestring=[b]"
}
% here you should include other packages with \usepackage

    \ifpdf

      % hyperref should be the last package loaded:
	  %\usepackage[pdftex]{hyperref}
      \usepackage{pst-pdf}
    \else

      % make the command \href from hyperref available as a 'print only'
      \newcommand{\href}[2]{#2}

    \fi

%Global Background must be put in preamble
\usebackgroundtemplate%
{%
    \includegraphics[width=\paperwidth,height=\paperheight]{Figuras/fundo.png}%
}
\setbeamertemplate{frametitle}[default][center]
 
\author{Othon Oliveira}
\title{Introdução do MySql}
\institute{SENAC - PROA} 
\date{} 
%\subject{} 

\begin{document}

\begin{frame}
\titlepage
%\date{}
\end{frame}

% Capa - requer o TikZ
\newcommand{\capa}{
    \begin{tikzpicture}[remember picture,overlay]
        \node at (current page.south west)
            {\begin{tikzpicture}[remember picture, overlay]
                \fill[shading=radial,top color=orange,bottom color=orange,middle color=yellow] (0,0) rectangle (\paperwidth,\paperheight);
            \end{tikzpicture}
          };
    \end{tikzpicture}
}


%+++++++++++++++++++++++ Telas iniciais  ---------------------------------
\begin{frame}\frametitle{Sumário}
\tableofcontents
\end{frame}
%---------------------------------------------------------------------

\begin{frame}
  \frametitle{Introdução ao MySql (e MariaDB)}
  \begin{itemize}
  \item Bem-vindo ao mundo do MySql, um sistema de gerenciamento de banco de dados SGBD!
  \pause
  \item   O MariaDB é SGBD e, é um fork ao MySql
  \end{itemize} 
\end{frame}



%+++++++++++++++++++++++++++++++++++++++++++++++
\section{Configuração Inicial}
%+++++++++++++++++++++++++++++++++++++++++++++++
\begin{frame}
  \frametitle{Configuração Inicial do MySQL}
  \begin{itemize}
    \item Criar um novo banco de dados: \texttt{CREATE DATABASE nome\_do\_banco;}
    \item Criar um novo usuário: \texttt{CREATE USER 'novo\_usuario'@'localhost' IDENTIFIED BY 'nova\_senha';}
    \item Conceder privilégios: \texttt{GRANT ALL PRIVILEGES ON nome\_do\_banco.* TO 'novo\_usuario'@'localhost';}
    \item Atualizar as alterações de privilégios: \texttt{FLUSH PRIVILEGES;}
  \end{itemize}
\end{frame}



%+++++++++++++++++++++++++++++++++++++++++++++++
\section{Linguagens de um Banco de Dados}
%+++++++++++++++++++++++++++++++++++++++++++++++

\begin{frame}
  \frametitle{Linguagens de Definição de Dados (DDL)}
  \begin{itemize}
    \item \texttt{CREATE TABLE}: Cria uma nova tabela.
    \item \texttt{ALTER TABLE}: Modifica uma tabela existente.
    \item \texttt{DROP TABLE}: Remove uma tabela.
    \item \texttt{CREATE INDEX}: Cria um novo índice.
  \end{itemize}
\end{frame}
%-------------------------------------------------------------
\begin{frame}
  \frametitle{Linguagens de Manipulação de Dados (DML)}
  \begin{itemize}
    \item \texttt{SELECT}: Recupera dados de uma ou mais tabelas.
    \item \texttt{INSERT INTO}: Insere novos registros em uma tabela.
    \item \texttt{UPDATE}: Modifica registros existentes em uma tabela.
    \item \texttt{DELETE FROM}: Remove registros de uma tabela.
  \end{itemize}
\end{frame}
%---------------------------------------------------------------
\begin{frame}
  \frametitle{Outros Dialetos e Comandos}
  \begin{itemize}
    \item \texttt{CREATE DATABASE}: Cria um novo banco de dados.
    \item \texttt{GRANT}: Concede privilégios aos usuários.
    \item \texttt{FLUSH PRIVILEGES}: Atualiza as alterações de privilégios.
    \item ... (outros comandos MariaDB, MySQL ou SQL específicos)
  \end{itemize}
\end{frame}
\end{document}

%+++++++++++++++++++++++++++++++++++++++++++++++
\section{Calculos com }
%+++++++++++++++++++++++++++++++++++++++++++++++

%-------------------------------------------------------------


%+++++++++++++++++++++++++++++++++++++++++++++++
\section{Calculos com }
%+++++++++++++++++++++++++++++++++++++++++++++++

%-------------------------------------------------------------

%-------------------------------------------------------------


%%+++++++++++++++++++++++++++++++++++++++++++++++


%+++++++++++++++++++++++++++++++++++++++++++++++




%-----------------------------------------------

\end{document}
